
%\section{Pos es tupla(x:Nat, y:Nat)}

%\section{Tipo es Bool}

%\section{Nombre es String}

\section{Base de Datos}

\subsection{Interfaz}

\sexc{Base}
$\textbf{usa}$  
\generos{nat, string, tabla, regisro, campo, dato}

%%%%%%%%%%%%%%%%%%%%%%%%%%%%%%%%%%%%%%%%%%%%%%%%%%%%%%%%%%%%%%%%%%%%%%%%%%%%

\subsubsection*{Operaciones}


%%%%%% observadores basicos %%%%%%

\operacion{tablas}{in b: base }{res : conj(string)}
 {true}
 {$res \igobs nombre(t)$}
 {Devuelve el nombre de la tabla ingresada por parametro.}
 {O($1$)}
 {Se retorna res por copia, por ser un tipo basico.}

\operacion{dameTabla}{in b: base }{res : tabla}
 {true}
 {$res \igobs claves(t)$}
 {Devuelve un conjunto de campos que son claves en la tabla ingresada por parametro.}
 {O($1$)}
 {Se devuelve un iterador al conjunto claves por referencia.}
 
\operacion{hayJoin?}{in t1: string, in t2: string, in t: base }{res : bool}
 {true}
 {$res \igobs indices(t)$}
 {Devuelve un conjunto de los indices de la tabla ingresada por parametro.}
 {O($calcular$)}
 {Se devuelve res por referencia y no es modificable.}
 
\operacion{campoJoin}{in t1: string, in t2: string, in t: base }{res : itConjTrie(campo)}
 {true}
 {$res \igobs campos(t)$}
 {Devuelve un conjunto a los campos de la tabla ingresada por parametro.}
 {O($1$)}
 {Se devuelve res por referencia.}

% %%%%%% generadores %%%%%%%
 
 \operacion{nuevaDB}{}{res : base}
 {$True$}
 {$res \igobs nuevaDB()$}
 {Crea una base sin tablas.}
 {O($calcular$)}
 {}

 \operacion{agregarTabla}{in t: tabla, in b: base}{}
 {$b\_0$=b $\land$ nombre(t)$\notin$ tablas(b) $\land$ Vacio?(t.registros)}
 {agregarTabla(t\,$b\_0$)}
 {Agrega una tabla a la base de datos.}
 {O($calcular$)}
 {Agrega tabla por referencia.} 
 
 \operacion{insertarEntrada}{in reg: registro, in t: string, in b: base}{}
 {$b\_0$=b $\land$ t$\in$tablas(b) $\yluego$ puedoInsertar?(dameTabla(t)\, reg)}
 {insertarEntrada(r\,t \,$b\_0$)}
 {Inserta el registro a la tabla que corresponde al string pasado por parametro.}
 {O($calcular$)}
 {} 
 
  \operacion{borrar}{in cr: registro, in t: string, in b: base}{}
 {$b\_0$=b $\land$ t$\in$tablas(b) $\land$ \#(cr.DiccClaves)}
 {borrar(cr\, t\, $b\_0$)}
 {Borra los registros que cumplan el criterio cr pasado por parametro.}
 {O($calcular$)}
 {} 	

  \operacion{generarVistaJoin}{in t1: string, in t2: string, in c: campo, in b: base}{}
 {b\_0=b $\land$ t1$\sqsupset$t2 $\land$ \{t1\, t2\}$\subseteq$tablas(b) $\yluego$ (c$\in$dameTabla(t1\, b).diccClaves $\land$ c$\in$dameTabla(t2\, b).diccClaves $\land$ $\neg$hayJoin?(t1\, t2\, b)}
 {generarVistaJoin(cr, t, $b\_0$)}
 {Borra los registros que cumplan el criterio cr pasado por parametro.}
 {O($calcular$)}
 {} 	
 
   \operacion{borrarJoin}{in t1: string, in t2: string, in b: base}{}
 {$b\_0$=b $\land$ hayJoin?(t1\, t2\, b)}
 {borrarJoin(t1\, t2\, $b\_0$)}
 {Borra correspondiente a los nombres de tablas, pasados por parametro.}
 {O($calcular$)}
 {}

%%%%%%% Otras Operaciones %%%%%%%%%%%

 \operacion{registros}{in t: string, in b: base}{res: conj(registro)}
 {t$\in$tablas(b)}
 {$res \igobs registros(t\, b)$}
 {Retorna el conjunto de registros correspondientes al nombre de tabla pasado por parametro}
 {O($calcular$)}
 {Se retorna el conjunto de registros por referencia.} 


 \operacion{vistaJoin}{in t1: string, in t2: string, in b: base}{res: conj(registro)}
 {\{t1\, t2\}$\subseteq$tablas(b) $\land$ hayJoin?(t1\,t2\,b)}
 {$res \igobs vistaJoin(t1\, t2\, b)$}
 {Retorna el conjunto de registros correspondientes al nombre de tabla pasado por parametro}
 {O($calcular$)}
 {Se retorna el conjunto de registros por referencia.} 

 \operacion{cantidadDeAccesos}{in t: string, in b: base}{res: nat}
 {t$\in$tablas(b)}
 {$res \igobs cantidadDeAccesos(t\,b)$}
 {Retorna la cantidad de modificaciones correspondientes al nombre de tabla pasado por parametro.}
 {O($calcular$)}
 {Se retorna res por referencia.} 

 \operacion{tablaMaxima}{in b: base}{res: string}
 {$\neg\emptyset$?(tablas(b))}
 {$res \igobs tablaMaxima(t\,b)$}
 {Retorna el nombre de la tabla con la mayor cantidad de modificaciones.}
 {O($calcular$)}
 {Se retorna el nombre de la tabla por referencia.} 
  
 \operacion{encontrarMaximo}{in t: string, in ct: conj(string), in b: base}{res: string}
 {\{t\}$\cup$ct $\subseteq$tablas(b)}
 {$res \igobs tablaMaxima(t\,b)$}
 {Retorna ...}
 {O($calcular$)}
 {Se retorna el nombre de la tabla por referencia.}
  
 \operacion{buscar}{in criterio: registro, in t: string, in b: base}{res: conj(registro)}
 {t$\in$tablas(b)}
 {$res \igobs tablaMaxima(t\,b)$}
 {Retorna ...}
 {O($calcular$)}
 {Se retorna el nombre de la tabla por referencia.}
  
 
 
%%%%%%%%%%%%%%%%%%%%%%%%%%%%%%%%%%%%%%%%%%%%%%%%%%%%%%%%%%%%%%%%%%%%%%%%%%%%%
\subsection{Representación}

\serc{Base}
{
\donde{estr}{
\tupla{TablaMaxima: Tmax, Tablas: DiccTrie(NombreTabla; info\_tabla)}}
		\donde{info\_tabla}{\tupla{TActual: tabla, Joins: DiccTrie(NombreTabla; info\_join)}}
      	\donde{info\_join}{\tupla{ 
      								Rcambios: Cola(DatoCambio),
      								campoJ: campo, 
      								campoT: tipo,
      								JoinS: ConjTrie(registro),
      								ValoresS: ConjTrie(String),
      								JoinN: ConjNat(registro),
      								ValoresN: ConjNat(Nat)}}
		\donde{Tmax}{\tupla{NomTabla: NombreTabla, $\#$Modif: Nat}}
		\donde{DatoCambio}{\tupla{Reg: Registro, NomOrigen: NombreTabla, Accion: Bool}}
}

\subsubsection*{Invariante de representación}

\begin{enumerate}
  \item El Nombre de la tabla es un String acotado.
  \item Indices es un arreglo de tamaño 2, que aloja el Indice correspondiente segun el orden de creacion.
  \item Para toda Dato que es clave en Indice, su significado llamemoslo sign esta incluido en Registros.
  \item 

\end{enumerate}


\subsubsection*{Función de abstracción}
%\abs{sistema}{CampusSeguro}{s}{cs}
%$s.campus \igobs campus(cs)$ $\land$ \\
%$s.estudiantes \igobs estudiantes(cs)$ $\land$ \\
%$s.hippies \igobs hippies(cs)$ $\land$ \\
%$s.agentes \igobs agentes(cs)$ $\land$ \\
%$((\paratodo{nombre}{n}) s.hippies.definido(n) \impluego s.hippies.obtener(n) \igobs posEstYHippie(n,cs)$ $\lor$ \\
%$(\paratodo{nombre}{n}) s.estudiantes.definido(n) \impluego s.estudiantes.obtener(n) \igobs posEstYHippie(n,cs))$ \\
%$(\paratodo{placa}{pl}) s.agentes.definido(pl) \impluego s.estudiantes.obtener(pl).pos \igobs posAgente(pl,cs))$ \\
%$(\paratodo{placa}{pl}) s.agentes.definido(pl) \impluego s.estudiantes.obtener(pl).cantSanciones \igobs cantSanciones(pl,cs))$ \\
%$(\paratodo{placa}{pl}) s.agentes.definido(pl) \impluego s.estudiantes.obtener(pl).cantCapturas \igobs cantCapturas(pl,cs))$ \\

%%%%%%%%%%%%%%%%%%%%%%%%%%%%%%%%%%%%%%%%%%%%%%%%%%%%%%%%%%%%%%%%%%%%%%%%%%%%%%%%%%
\newpage

\subsection{Algoritmos}

\algoritmo{tablas}{in b: estr}{res: ConjTrie(string)}{
  \State $res \larr b.tablas.DiccClaves $ \complejidad{O(1)}
}{O(1)}

\algoritmo{hayJoin?}{in t1: string, in t2: string, in b: estr}{res: bool}{
	\State   res $\larr$Definido?(Obtener(b, t1).Joins,t2) $\larr$ Definido?(Obtener(b, t2).Joins,t1)
}{O(1)}

\algoritmo{campoJoin}{in t1: string, in t2: string, in b: estr}{res: campo}{
		\State res$\larr$ Obtener(Obtener(b, t1).Joins, t2).campoJ
}{O(1)}

%%%%%%%%%%%%% generadores %%%%%%%%%%%%%
\algoritmo{nuevaDB}{}{res: estr)}{ 
	\State String s
	\State Nat n $\larr$ 0
  	\State res $\larr$ $\langle \langle s, 0 \rangle, vacio()\rangle$  \complejidad{O(1)} 
}{O(1)}

\algoritmo{agregarTabla}{in t: tabla, in/out b: estr}{}{
	\State	info\_tabla $\larr$ $\langle t.cantidadDeAccesos, t, vacio()\rangle$  \complejidad{O(1)}
	\State	Definir(b.tablas, nombre(t), info\_tabla)								 \complejidad{O(1)}
}{O(1)}

\algoritmo{insertarEntrada}{in reg: registro, in t: string, in/out b: estr}{}{
	\State	Obtenemos la tabla es O(1) porque su nombre esta acotado.
	\State	info\_tabla infoT $\larr$	Obtener(b.tablas, t).TActual		\complejidad{O(1) Por referencia}
	\State	Agrego el registro a la tabla. \complejidad{O(1)}
	\State	Tabla T $\larr$ infoT.TActual \complejidad{O(1)}
	\State	agregarRegistro(reg, T)								\complejidad{O(1)}
	\State	Ahora si hay Joins actualizo la informacion temporal de cada Join.
	\If{$\neg\emptyset$?(infoT.Joins)} \complejidad{O(1)}
				\State	ItConjString(String) itNomTab $\larr$ CrearIt(Claves(infoT.Joins))\complejidad{O(1)}
				\While{HaySiguiente?(itNomTab)}								\complejidad{O(1)}
					\State	info\_join infoJ $\larr$ Obtener(infoT.Joins, Siguiente(NomTab)) \complejidad{O(1)}
					\State	Encolar(infoJ.Rcambios, $\langle reg, Siguiente(NomTab),  true\rangle$) \complejidad{O(L)}
					\State	Encolar es O(L) porque se copia un registro con su cantidad de campos acotada
					\State	y los valores string copiarlos tiene costo O(L), siendo L el valor string mas largo.
					\State	Avanzar(itClaves)\complejidad{O(1)}
				\EndWhile
	\EndIf
	\If{CantidadDeAccesos(T) $\>$b.TablaMaxima.\#Modif}
			\State	b.TablaMaxima.NomTabla $\larr$ Copiar(Nombre(T)) \complejidad{O(L)}
			\State	b.TablaMaxima.\#Modif $\larr$ Copiar(CantidadDeAccesos(T))\complejidad{O(1)}
	\EndIf
}{O(T*L + Log(n)) siendo n la cantidad de registros en t.}

\algoritmo{borrar}{in cr: registro, in t: string, in/out b: estr}{}{
	\State	info\_tabla infoT $\larr$	Obtener(b.tablas, t).TActual		\complejidad{O(1) por ref}
	\State	Tabla T $\larr$ infoT.TActual \complejidad{O(1) por ref}
	\State	NombreTabla NomTab $\larr$ NombreTabla(T)
	\State	Actualizo la eliminacion en los joins, que por cada tabla con la que se tenga 
	\State	join por dicho campo, en ese caso el campo pasado
	\State	por parametro es campo clave de la tabla t.
	\If{$\neg\emptyset$?(Claves(infoT.Joins))} \complejidad{O(1)}
			\State	Creo el iterador, para navegar los nombres de tablas con los que tiene Join
			\State	itNom $\larr$ CrearIt(Claves(infoT.Joins)) \complejidad{O(1)}
			\While{HaySiguiente?(itNom)} \complejidad{O(Cant de tablas)}
					\State info\_join infoJ $\larr$ Siguiente(itNom) \complejidad{O(1)}
					\State Verifico si el Join esta creado en base al 
					\State campo del cr pasado por parametro.
					\State 
					\If{infoJ.campoJ=DameUno(Campos(cr))} \complejidad{O(1)}
							\State	Entonces solo actualizo la cola temporal del join
							\State	Registro reg $\larr$ Buscar(cr, T) \complejidad{O(L + Log(n))}
							\State	Encolar(infoJ.Rcambios, $\langle reg, Siguiente(itNom),  False\rangle$)\complejidad{O(L)}
					\Else
							\State	Si el campo del criterio de borrado, es distinto que el campo del Join.
							\State	Primero busco que registros coinciden con el criterio
							\State	en el peor caso con complejidad O(cantidad de registros de t) sin indices.
							\State	Conj(Registro) cjc $\larr$ Buscar(cr, T)\complejidad{O(L + Log(n))}% Coincidencias(cr, Registros(r))
							\State	ItConj(Registro) itReg $\larr$ CrearIt(cjc)
							\State	Luego agrego estos registros a la cola temporal de cambios del join
							\While{HaySiguiente?(itReg)}
									\State	Encolar(infoJ.Rcambios, $\langle Siguiente(itReg), NomTab,  False\rangle$) \complejidad{O(L)}
									\State Avanzar(itReg)
							\EndWhile
					\EndIf
			\EndWhile
	\EndIf
	\State	Habiendo eliminados los registros que cumplen con el criterio de borrado
	\State	Elimino de la tabla los registros que cumplen el criterio de borrado.
	\State	Siendo n la cantidad de registros de t y T la cantidad de tablas en la base
	\State	En peor caso con costo O(n)
	\State	borrarRegistro(r, $T\_$actual)						\complejidad{O(T*L + n)}
	
	\If{CantidadDeAccesos(T) $\>$b.TablaMaxima.\#Modif}
			\State	b.TablaMaxima.NomTabla $\larr$ Copiar(Nombre(T)) \complejidad{O(L)}
			\State	b.TablaMaxima.\#Modif $\larr$ Copiar(CantidadDeAccesos(T))\complejidad{O(1)}
	\EndIf
}{O(1)}

\algoritmo{generarVistaJoin}{in t1: string, in t2: string, in c: campo, in/out b: estr}{}{
	\State	Join $\larr$ vacio()
	\State	T\_actual1 $\larr$	Obtener(b.tablas, t1).Tactual		\complejidad{O(1)}
	\State	T\_actual2 $\larr$	Obtener(b.tablas, t2).Tactual		\complejidad{O(1)}
	% Vemos si existe un indice en ambas tablas para c
	\If{Pertenece?(Indices(T\_actual1), c) $\land$ Pertenece?(Indices(T\_actual1), c)} \complejidad{O(1)}
			\State	ind1 $\larr$ Obtener(T\_actual1.Indices, c)  \complejidad{O(calcular)}
			\If{tipoCampo(T\_actual1, c)}
					\State	ConjNat(Nat) cjNat $\larr$ vacio()
					\State	itvalores $\larr$ CrearItConjNat(ind1.PorNat.DiccClaves)
					\While{HaySiguiente(itvalores)}
							\State	Registro r $\larr$ Obtener(ind1.PorNat, Siguiente(itvalores))
							\State	Nat n $\larr$ ValorNat(Obtener(r, c))
							\If{$\neg$ Pertenece?(cjNat, n)}
									\State	AgregarRapido(cjNat, r)
							\EndIf
							\State	Avanzar(itvalores)
					\EndWhile
					\State	ind2 $\larr$ Obtener(T\_actual2.Indices, c)		
					\State	itvalores $\larr$ CrearItConjNat(ind2.PorNat.DiccClaves)
					\While{HaySiguiente(itvalores)}
							\State	Registro r $\larr$ Obtener(ind2.PorNat, Siguiente(itvalores))
							\State	Nat n $\larr$ ValorNat(Obtener(r, c))
							\If{$\neg$ Pertenece?(cjNat, n)}
									\State	AgregarRapido(cjNat, r)
							\EndIf
							\State	Avanzar(itvalores)
					\EndWhile
%					\While{HaySiguiente(itvalores)}
%							\State	r1 $\larr$ Obtener(ind1.PorNat, Siguiente(itvalores))
%							\State	r2	$\larr$ Obtener(ind2.PorNat, Siguiente(itvalores))
%							\State	cj1 $\larr$ AgregarRapido(vacio(), r1)
%							\State	cj2 $\larr$ AgregarRapido(vacio(), r2)
%							\State	nuevor $\larr$ combinarRegistros(c, cj1, cj2)
%							\State	AgregarRapido(Join,DameUno(nuevor))
%							\State	Avanzar(itvalores)
%					\EndWhile
			\Else
					\State	itvalores $\larr$ CrearItConjString(ind1.PorString.DiccClaves)
					\While{HaySiguiente(itvalores)}
							\State	r1 $\larr$ Obtener(ind1.PorString, Siguiente(itvalores))
							\State	r2	$\larr$ Obtener(ind2.PorString, Siguiente(itvalores))
							\State	cj1 $\larr$ AgregarRapido(vacio(), r1)
							\State	cj2 $\larr$ AgregarRapido(vacio(), r2)
							\State	nuevor $\larr$ combinarRegistros(c, cj1, cj2)
							\State	AgregarRapido(Join, DameUno(nuevor))
							\State	Avanzar(itvalores)
					\EndWhile
			\EndIf
	\Else
			\State	cjr1 $\larr$ T\_actual1.registros
			\State	cjr2 $\larr$ T\_actual1.registros
			\State	Join$\larr$combinarRegistros(c, cjr1,cjr2)
	\EndIf
	\State info\_join $\larr$ $\langle 0, vacio(), vacio(), c, tipoCampo(T\_actual1, c), Join\rangle$
	\State Definir(b.Joins,	$\langle t1, t2\rangle$ , info\_join)

}{O(1)}

\algoritmo{borrarJoin}{in t1: string, in t2: string, in/out b: estr}{}{
	\If{Pertenece?(b.Joins, $\langle t1, t2\rangle$  )}
			\State	Borrar(b.Joins, $\langle t1, t2\rangle$ )
	\Else
			\If{Pertenece?(b.Joins, $\langle t2, t1\rangle$ )}
					\State	Borrar(b.Joins, $\langle t2, t1\rangle$)
			\EndIf
	\EndIf
}{O(1)}



%\algoritmo{campos}{in t:tab}{res:itConjTrie(campo)}{
%  \State $res \larr CrearItConjTrie(t.Campos.ClavesDicc)$ \complejidad{O(1)}
%}{O(1)} % en otras operaciones tiene q figurar ese conjunto
%
%\algoritmo{tipoCampo}{in c:campo, in t:tab}{res:Tipo}{
%  \State $res \larr Significado(t.Campos,c)$ \complejidad{O(1)}
%}{O(1)}
%
% \algoritmo{registros}{in t:tab}{res:itConj(registro)}{
%    \State $res \larr CrearItConj( t.registros )$ \complejidad{$\theta$(L+ log(n))}
% }{$\theta$(L+ log(n))}
%
%  \algoritmo{cantDeAccesos}{in t:tab}{res:nat}{
%    \State $res \larr t.cantDeAccesos$ \complejidad{$\theta$(1)}
% }{$\theta$(1)}
%
%  \algoritmo{puedoInsertar?}{in r:registro, in t:tab}{res:bool}{
%    \State res $\larr$ campatible(r,t) $\land$ $\neg$hayCoincidencia(r,claves(r),registros(t))  \complejidad{$\theta$(L+log(n))}
% }{$\theta$(L+log(n))}
%
%
%  \algoritmo{compatible}{in r:registro, int t:tab}{res:bool}{
%      \State res $\larr$ campatible(r,t) $\yluego$ mismosTipos(r,t)  \complejidad{$\theta$(1)}
% }{O($1$)}
%
%
%  \algoritmo{puedeIndexar}{in c:campo, in t:tab}{res:bool}{
%      \State res  $\larr $ Definido?(t.campos,c) $\yluego \neg$ Esta?(c,t.Indices.IndicesL) $\land$ \\ (long(t.Indices.IndicesL)=0 $\lor$ (long(t.Indices.IndicesL)$<$2 $\land$ \\ (tipoCampo(c,t)!=tipoCampo(t.Indices.IndicesL.primero,t))))    \complejidad{$\theta$(1)} %funcion axuliar hayIndiceC?(c,t)
% }{O(calcular)}
% 
%
%  \algoritmo{combinarRegistros}{in c:campo, in cr1:Conj(registro), in cr2:Conj(registro)}{res:Conj(registros)}{
%  	\State	res $\larr$ vacio();
%	\State	itcr1 $\larr$ CrearItConjTrie(cr1)
%  	\State 	\While{HaySiguiente(itcr1)}
%				\State	AgregarRapido(res, combinarTodos(c,Siguiente(itcr1),cr2));	%combinarTodos deberia hacer por copia asi no afecta el original.
%				\State	Avanzar(itcr1);
%	\State 	\EndWhile   
% }{O(calcular)}
% 
%   \algoritmo{hayCoincidencia}{in r:registro, in cc:ConjTrie(campo), in cr:Conj(registro)}{res:bool}{
%  	\State	itcr $\larr$ CrearItConj(cr);
%    \State	res $\larr$ false;
%  	\State 	\While{HaySiguiente(itcr)}
%				\State	res $\larr$ 	coincideAlguno(r,cc,Siguiente(itcr)) $\lor$res;
%				\State	Avanzar(itcr);
%	\State 	\EndWhile   
% }{O(calcular)}
%
% \algoritmo{coincidencias}{in crit:registro, in cr:Conj(registro)}{res:itConj(registro)}{
%  	\State 	res $\larr$ CrearItConj(vacio());
%  	\State 	\While{HaySiguiente(cr)}
%				\State	\If {coincidenTodos(crit,campos(crit),Siguiente(cr))} 
%								\State	AgregarAtras(res,Siguiente) 
%				\State	\EndIf				
%				\State	Avanzar(cr);
%	\State	\EndWhile   
% }{O(calcular)}
% 
%  \algoritmo{minimo}{in c:campo, in t:tab}{res:dato}{
%  	\State 	res $\larr$ min( dameColumna(  c, t.registros  );
% }{O(calcular)}
% 
%   \algoritmo{maximo}{in c:campo, in t:tab}{res:dato}{
%  	\State 	res $\larr$ max( dameColumna(  c,  t.registros  );
% }{O(calcular)}
% 
%    \algoritmo{dameColumna}{in c:campo, in cr:Conj(registro)}{res:itConj(dato)}{
%  	\State 	itcr $\larr$ CrearItConj(cr);
%  	\State 	res $\larr$ CrearItConj(vacio());
%  	\State 	\While{HaySiguiente(itcr)}
%				\State	\If {Definido?(Siguiente(itcr),c)} 
%								\State	AgregarAtras(res,Obtener(Siguiente(itcr), c)) ;
%				\State	\EndIf				
%				\State	Avanzar(itcr);
%	\State	\EndWhile     	
% }{O(calcular)}
% 
%    \algoritmo{mismosTipos}{in r:registro, in t:tab}{res:bool}{
%  	\State 	res $\larr$ True;
%  	\State 	itconjClaves $\larr$ CrearItConj(claves(r)); 
%  	\State 	\While{HaySiguiente(itconjClaves)}
%					\State	res $\larr$ res$\land$(tipo?(Obtener(r,Siguiente(itconjClaves))) = tipoCampo(Siguiente(itconjClaves)),t)	;
%					\State	Avanzar(cr);
%	\State	\EndWhile     	
% }{O(calcular)}
%
%
%  %%%%%%%%%%%%%%%%%%%%%%%%%%%%%%%%%%%%%%%%%%%%%%%%%%%%%%%%%%%%%%%%%%%%%%%%%%%%%%%%%
%
%\subsection{Algoritmos operaciones auxiliares}
%\algoritmo{agregarEstudiante}{in/out campus:campusSeguro, in pos:pos, in nombre:nombre}{}{
%    \State $campus.campus[pos.x][pos.y].hayEst \larr True$ \complejidad{O(1)}
%    \State $campus.campus[pos.x][pos.y].estudiante \larr definir(campus.estudiantes, nombre, pos)$ \complejidad{O(long(nombre))}
%}{O(long(nombre))}
%
%\algoritmo{agregarHippie}{in/out campus:campusSeguro, in pos:pos, in nombre:nombre}{}{
%    \State $campus.campus[pos.x][pos.y].hayHippie \larr True$ \complejidad{O(1)}
%    \State $campus.campus[pos.x][pos.y].hippie \larr definir(campus.hippies, nombre, pos)$ \complejidad{O(long(nombre))}
%}{O(long(nombre))}
%
%
% \algoritmo{sancionarAgentesVecinos}{in/out campus:campusSeguro, in pos:pos}{}{
%    % Sancionar VECINOS
%    \State $vecinos \larr campus.campusEstatico.vecinos(pos)$ \complejidad{O(1)}
%    \If {$campus.atrapadoPorAgente?(pos)$} 
%       \While {$i < vecinos.tamanio()$}  \complejidad{O(1)}
%           \If {$campus.campus[vecinos[i].x][vecinos[i].y].hayAgente?$}
%               \State $campus.sancionarAgente(vecinos[i].agente)$ \complejidad{O(1)}
%           \EndIf
%           \State $i++$
%       \EndWhile
%    \EndIf
% }{O(1)}
%
%\algoritmo{sancionarAgentesEncerrandoEstVecinos}{in/out campus:campusSeguro, in pos:pos}{}{
%   \State $vecinos \larr campus.campusEstatico.vecinos(pos)$ \complejidad{O(1)}
%   \State $i \larr 0$
%   \While {$i < vecinos.tamanio$} \complejidad{O(1)}
%      \If {$campus.campus[vecinos[i].x][vecinos[i].y].hayEst$ $\land$ $atrapadoPorAgente?(campus,pos)$}  \complejidad{O(1)}
%          \State $sancionarAgentesVecinos(campus,pos)$ \complejidad{O(1)}
%      \EndIf
%      \State $i++$
%    \EndWhile
%}{O(1)}
%
%\algoritmo{sancionarAgente}{in/out campus:campusSeguro, in/out agente:itDiccRapido}{}{
%    \State $campus.conKSanciones.ocurrioSancion \larr True$ \complejidad{O(1)}
%    \State $agente.siguiente.cantSanciones + 1$ \complejidad{O(1)}
%    \State $agente.siguiente.miUbicacion.eliminarSiguiente()$ \complejidad{O(1)}
%    \State // El iterador mismas apunta a la posicion correspondiente del agente dentro de la lista ordenada por cantSanciones
%    \State // Como la lista en el peor caso puede contener a todos los agentes con igual cant de sanciones
%    \State // la mayor cantidad posible de iteraciones del ciclo es 4
%    \While {$agente.siguiente.mismcampus.haySiguiente()$ \\$\land$ $agente.siguiente.mismas.siguiente.cantSanciones < agente.siguiente.cantSanciones$}
%      \State $agente.siguiente.mismas.avanzar()$ \complejidad{O(1)}
%    \EndWhile
%    \State // Si no hay siguiente o  si la cantidad de sanciones del siguiente es menor que la del agente, entonces, 
%    \State // creo un conMismasBucket, lo inserto como siguiente y me guardo el iterador en miUbicacion
%    \State // Sino, agrego el agente al conj de agentes del siguiente y me guardo el iterador en miUbicacion
%    \If {$\neg(agente.siguiente.mismas.haySiguiente)$ $\lor$
%        \\$(agente.siguiente.mismas.haySiguiente \land$\\$agente.siguiente.cantSanciones=agente.siguiente.mismas.cantSanciones)$} \complejidad{O(1)}
%      \State $nConMismasB \larr nuevaTupla(CrearNuevoDiccLineal(), agente.siguiente.cantSanciones)$
%      \State $agente.siguiente.mismas \larr agente.siguiente.mismas.agregarComoSiguiente(nConMismasB)$ \complejidad{O(1)}
%      \State $agente.siguiente.miUbicacion$ $\larr$
%      \State $agente.siguiente.mismas.siguiente.agentes.agregarComoSiguiente(agente.siguiente.pl)$ \complejidad{O(1)}
%    \Else 
%      \State $agente.siguiente.mismas.siguiente.agentes.agregarComoSiguiente(agente.siguiente.pl)$ \complejidad{O(1)}
%    \EndIf
%}{O(1)}
%
% \algoritmo{atrapadoPorAgente?}{in campus:campusSeguro, in pos:pos}{res:bool}{
%    \State $vecinos \larr campus.campusEstatico.vecinos(pos)$
%    \State $alMenos1Agente \larr False$ \complejidad{O(1)}
%    \State $i \larr 0$
%    \If {$\neg(encerrado?(pos, campus.campusEstatico.vecinos(pos)))$}
%      \State $return$ $false$
%    \EndIf
%    \State // Veo si hay algun agente alrededor
%     \While {$i<vecinos.tamanio()$} \complejidad{O(1)}
%       \If {$as.campus[vecinos[i].x][vecinos[i].y].hayAgente?$}
%          \State $return$ $true$            
%        \EndIf
%        \State $i++$ \complejidad{O(1)}
%    \EndWhile
% }{O(1)}
%
% \algoritmo{hippificarEstudiantesVecinos}{in/out campus:campusSeguro, in pos:pos}{}{
%    \State $vecinos \larr campus.campusEstatico.vecinos(pos)$ \complejidad{O(1)}
%    \State $i \larr 0$ \complejidad{O(1)}
%    \While {$i<vecinos.tamanio()$} \complejidad{O(long(nombre))}
%        \If {$estAHippie?(campus,vecinos[i])$}
%            \State $hippificar(campus,vecinos[i])$ \complejidad{O(long(nombre))}
%        \EndIf
%        \State $i++$ \complejidad{O(1)}
%    \EndWhile
% }{O(long(nombre))}
%
%\algoritmo{hippificar}{in/out campus:campusSeguro, in pos:pos}{}{
%    \State // PRE: La posicion esta en el tablero y hay estudiante en la posicion
%    \State $as.campus[pos.x][pos.y].hayHippie \larr True$ \complejidad{O(1)}
%    \State $as.campus[pos.x][pos.y].hippie.agregarComoSiguiente(nombre, pos)$ \complejidad{O(long(nombreEstudiante))}
%    \State $as.campus[pos.x][pos.y].hayEst \larr False$ \complejidad{O(1)}
%    \State $as.campus[pos.x][pos.y].estudiante.eliminarSiguiente()$ \complejidad{O(long(nombreEstudiante))}
%}{O(long(nombre))}
%
% \algoritmo{estAHippie?}{in campus:campusSeguro, in pos:pos}{res:bool}{
%    \If {$\neg(encerrado?(pos,vecinos))$}
%      \State $return$ $false$ \complejidad{O(1)}
%    \EndIf
%    \State $i$ $\larr$ $0$ \complejidad{O(1)}
%    \State $cantHippies$ $\larr$ $0$ \complejidad{O(1)}
%    \State $vecinos \larr campus.campusEstatico.vecinos(pos)$ \complejidad{O(1)}
%    \While {$i < vecinos.tamanio()$}
%      \If {$campus[vecinos[i].x][vecinos[i].y].hayHippie$}
%        \State $cantHippies++$ \complejidad{O(1)}
%      \EndIf
%      \State $i++$
%    \EndWhile
%    \State $return$ $cantHippies \ge 2$ \complejidad{O(1)}
% }{O(1)}
%
%\algoritmo{hippieAEst?}{in campus:campusSeguro, in pos:pos}{res:bool}{
%  \State $i$ $\larr$ $0$ \complejidad{O(1)}
%  \State $vecinos \larr campus.campusEstatico.vecinos(pos)$ \complejidad{O(1)}
%  \While {$i < vecinos.tamanio()$} \complejidad{O(1)}
%      \If {$\neg(as.campus[vecinos[i].x][vecinos[i].y].hayEst?)$}
%        \State $return$ $False$ \complejidad{O(1)}
%      \EndIf
%  \EndWhile
%  \State $return$ $True$
%}{O(1)}
%
% \algoritmo{encerrado?}{in campus:campusSeguro, in pos:pos}{}{
%    \State $vecinos \larr vecinos(as.campusEstatico, pos)$ \complejidad{O(1)}
%    \State $i \larr vecinos.tamanio()$ \complejidad{O(1)}
%    \While {$i<vecinos.tamanio()$} \complejidad{O(1)}
%       \If {$\neg(campus.campus[vecinos[i].x][vecinos[i].y].hayAgente?$ $\lor$
%             \\$campus.campus[vecinos[i].x][vecinos[i].y].hayEst?$ $\lor$
%             \\$campus.campus[vecinos[i].x][vecinos[i].y].hayHippie?$ $\lor$
%             \\$campus.campus[vecinos[i].x][vecinos[i].y].hayObst?)$} \complejidad{O(1)}
%              \State $return$ $false$ \complejidad{O(1)}
%        \EndIf
%        \State $i++$ \complejidad{O(1)}
%    \EndWhile
%    \State $return true$
% }{O(1)}
%
% \algoritmo{aplicarHippiesVecinos}{in/out campus:campusSeguro, in pos:pos}{}{
%    \State $vecinos \larr campus.campusEstatico.vecinos(pos)$ \complejidad{O(1)}
%    \State $i \larr 0$ \complejidad{O(1)}
%    \While {$i < vecinos.tamanio()$} \complejidad{O(long(nombre))}
%       \State aplicarHippie(campus, pos) \complejidad{O(long(nombre))}
%    \EndWhile
% }{O(long(nombre))}
%
% \algoritmo{aplicarHippie}{in/out campus:campusSeguro, in pos:pos}{}{
%    \State // PRE: pos valida y hayHippie en campus.campus[pos.x][pos.y]
%    \If {$campus.campus[pos.x][pos.y].hayHippie$}
%      \If {$as.hippieAEst(pos)$} \complejidad{O(1)}
%          \State $campus:campusSeguro.campus[pos.x][pos.y].hayHippie \larr False$ \complejidad{O(1)}
%          \State $campus:campusSeguro.campus[pos.x][pos.y].hayEst \larr True$ \complejidad{O(1)}
%          \State $as.campus[pos.x][pos.y].estudiante \larr CrearIt(campus.hippies)$ \complejidad{O(1)}
%          \State $campus.campus[pos.x][pos.y].estudiante.$
%          \State $agregarComoSiguiente(campus.campus[pos.x][pos.y].estudiante.nombre)$ \complejidad{O(long(nombre))}
%          \State $campus.campus[pos.x][pos.y].hippie.eliminarSiguiente()$ \complejidad{O(long(nombre))}
%      \Else
%        \If {$campus.campus[pos.x][pos.y].hayHippie?$ $\land$ $atrapadoPorAgente(pos)$}
%            \State $vecinos \larr campus.campusSeguro.vecinos(pos)$ \complejidad{O(1)}
%            \State $i \larr 0$ \complejidad{O(1)}
%            \While {$i < vecinos.tamanio()$} \complejidad{O(1)}
%                \State $posAct \larr vecinos[pos.x][pos.y]$ \complejidad{O(1)}
%                \State $info$ $\larr$ $campus.campus[vecinos[i].x][vecinos[i].y]$ \complejidad{O(1)}
%                \If {$posAct.hayAgente$}
%                    \State $info.agente.siguiente.cantCapturas++$ \complejidad{O(1)}
%                    \State // Actualizar mas vigilante
%                    \If {$campus.masVigilante.siguienteSignificado().cantCapturas < $\\$info.agente.siguienteSignificado().cantCapturas$}
%                        \State $campus.masVigilante \larr info.agente$ \complejidad{O(1)}
%                    \Else
%                        \If{$campus.masVigilante.siguienteSignificado().cantCapturas = $\\$info.agente.siguienteSignificado().cantCapturas$\\
%                            $\land campus.masVigilante.siguienteClave() < info.agente.siguienteClave()$} \complejidad{O(1)}
%                              \State $campus.masVigilante \larr info.agente$ \complejidad{O(1)}
%                        \EndIf
%                    \EndIf
%                \EndIf
%                \State $i++$
%            \EndWhile
%            \State $campus.campus[pos.x][pos.y].hayHippie? = False$ \complejidad{O(1)}
%            \State $campus.campus[pos.x][pos.y].hippie.eliminarSiguiente()$ \complejidad{O(long(nombre))}
%        \EndIf
%      \EndIf
%    \EndIf
% }{O(long(nombre))}
%
% \algoritmo{proxPosHippie}{in/out campus:campusSeguro, in nombre:string}{res:pos}{
%    \State // PRE: El nombre es un hippie y el hippie no esta encerrado
%    \State $posHippie \larr campus.hippies.obtener(nombre)$ \complejidad{O(long(nombre))}
%    \If {$campus.estudiantes.tamanio()>0$}
%      \State // Retorna de las posiciones mas cercanas, la que esta mas cerca del (0,0)
%      \State $proxPos \larr aPosMasCercana(campus.campusEstatico,posHippie,campus.estudiantes.significados)$ \complejidad{O($N_e$)}
%    \Else
%        \State // Retorna el ingreso mas cercan, en caso de empate, el de abajo
%        \State $proxPos \larr aIngresoMasCercano(campus.campusEstatico,posHippie)$ \complejidad{$O(1)$}
%    \EndIf
%    \State $res \larr proxPos$ \complejidad{O(1)}
% }{O($N_e$)}
%
%  \algoritmo{proxPosAgente}{in/out campus:campusSeguro, in posAgente:pos}{res:pos}{
%    \State // PRE: En la posicion hay un agente que se puede mover
%    \If {$campus.hippies.tamanio()>0$}
%      \State // Retorna de las posiciones mas cercanas, la que esta mas cerca del (0,0)
%      \State $proxPos \larr aPosMasCercana(campus.campusEstatico,posAgente,campus.hippies.significados)$ \complejidad{O($N_h$)}
%    \Else
%        \State // Retorna el ingreso mas cercano, en caso de empate, el de abajo
%        \State $proxPos \larr aIngresoMasCercano(campus.campusEstatico,posAgente)$ \complejidad{$O(1)$}
%    \EndIf
%    \State $res \larr proxPos$ \complejidad{O(1)}
% }{O($N_h$)}
%
%
% \algoritmo{aIngresoMasCercano}{in p:pos, cs: campusSeguro}{res:pos}{
%	\If{$p.Y\leq c.alto/2$}
%		\If{$PosValida(cs.campus,<p.X,p.Y-1>)\land \lnot HayAlgo(cs,<p.X,p.Y-1>)$}
%			\State $res \larr <p.X,p.Y-1>$
%		\Else
%			\If{$PosValida/c,<p.X+1,p.Y>)\land \lnot HayAlgo(c,<p.X+1,p.Y>)$}
%				\State $res \larr <p.X+1,p.Y>$
%			\Else
%				\If{$PosValida/c,<p.X-1,p.Y>)\land \lnot HayAlgo(c,<p.X-1,p.Y>)$}
%					\State $res \larr <p.X-1,p.Y>$
%				\Else
%					\State $res \larr <p.X,p.Y+1>$
%				\EndIf
%			\EndIf
%		\EndIf
%	\Else
%		\If{$PosValida(cs.campus,<p.X,p.Y+1>)\land \lnot HayAlgo(cs,<p.X,p.Y+1>)$}
%			\State $res \larr <p.X,p.Y-1>$
%		\Else
%			\If{$PosValida/c,<p.X+1,p.Y>)\land \lnot HayAlgo(c,<p.X+1,p.Y>)$}
%				\State $res \larr <p.X+1,p.Y>$
%			\Else
%				\If{$PosValida/c,<p.X-1,p.Y>)\land \lnot HayAlgo(c,<p.X-1,p.Y>)$}
%					\State $res \larr <p.X-1,p.Y>$
%				\Else
%					\State $res \larr <p.X,p.Y-1>$
%				\EndIf
%			\EndIf
%		\EndIf
%	\EndIf
%}
%{O($1$)}
%
% \algoritmo{ibusquedaBinariaPorSanciones}{in ar:arreglo(val:nat otr: $\alpha$>), in sanc:nat}{res:$<\alpha , bool>$)}{
% 	\State $res.\pi_2 \larr false$
%	\State $min \larr 0$
%	\State $max \larr |ar|$
%    \While {$max-min>1$} \complejidad{O(log($|ar|$))}
%       \State $med \larr (max-min)/2$
%       \If {$ar[med].val \leq sanc$}
%       		\State $min \larr med$
%       \Else
%       		\State $max \larr med$
%       	\EndIf
%    \EndWhile
%    \If {$ar[min].val = sanc$}
%    	\State $res \larr <ar[min].otr, true>$
%    \EndIf
% }{O(log($|ar|$)}