
%\section{Pos es tupla(x:Nat, y:Nat)}

%\section{Tipo es Bool}

\section{NombreTabla es String}

\section{Base de Datos}

\subsection{Interfaz}

\sexc{BaseDeDatos, nat, string,} {tabla, registro, campo, dato,}{ DiccString(String, alfha), DiccNat(Nat, beta)}

\generos{base}
%%%%%%%%%%%%%%%%%%%%%%%%%%%%%%%%%%%%%%%%%%%%%%%%%%%%%%%%%%%%%%%%%%%%%%%%%%%%

\subsubsection*{Operaciones}


%%%%%% observadores basicos %%%%%%
%
\operacion{tablas}{in b: base }{res : ItConjString(NombreTabla)}
 {true}
 {$res \igobs nombre(t)$}
 {Devuelve el nombre de la tabla ingresada por parametro.}
 {O($1$)}
 {Se retorna res por copia, por ser un tipo basico.}

\operacion{dameTabla}{in t: string, in b: base }{res : tabla}
 {true}
 {$res \igobs dameTabla(t,b)$}
 {Devuelve la tabla correspondiente al nombre ingresado por parametro.}
 {O($1$)}
 {Se devuelve la tabla correspondiente por referencia.}

\operacion{hayJoin?}{in t1: string, in t2: string, in t: base }{res : bool}
 {true}
 {$res \igobs indices(t)$}
 {Devuelve un conjunto de los indices de la tabla ingresada por parametro.}
 {O($1$)}
 {Se devuelve res por referencia y no es modificable.}
 
\operacion{campoJoin}{in t1: string, in t2: string, in t: base }{res : itConjTrie(campo)}
 {true}
 {$res \igobs campos(t)$}
 {Devuelve un conjunto a los campos de la tabla ingresada por parametro.}
 {O($1$)}
 {Se devuelve res por referencia.}

% %%%%%% generadores %%%%%%%
 
 \operacion{nuevaDB}{}{res : base}
 {$True$}
 {$res \igobs nuevaDB()$}
 {Crea una base sin tablas.}
 {O($1$)}
 {}

 \operacion{agregarTabla}{in t: tabla, in b: base}{}
 {$b\_0$=b $\land$ nombre(t)$\notin$ tablas(b) $\land$ Vacio?(t.registros)}
 {agregarTabla(t\,$b\_0$)}
 {Agrega una tabla a la base de datos.}
 {O($1$)}
 {Agrega tabla por referencia.} 
 
 \operacion{insertarEntrada}{in reg: registro, in t: string, in b: base}{}
 {$b\_0$=b $\land$ t$\in$tablas(b) $\yluego$ puedoInsertar?(dameTabla(t)\, reg)}
 {insertarEntrada(r\,t \,$b\_0$)}
 {Inserta el registro a la tabla t.}
 {O($T$)}
 {} 
 
  \operacion{borrar}{in cr: registro, in t: string, in b: base}{}
 {$b\_0$=b $\land$ t$\in$tablas(b) $\land$ \#(DiccClaves(cr))=1}
 {borrar(cr\, t\, $b\_0$)}
 {Borra los registros que cumplan el criterio cr pasado por parametro.}
 {O($T + Log(n)$)}
 {} 	

  \operacion{generarVistaJoin}{in t1: string, in t2: string, in c: campo, in b: base}{}
 {b\_0=b $\land$ t1$\sqsupset$t2 $\land$ \{t1, t2\}$\subseteq$tablas(b) $\yluego$ \\Pertenece?(Campos(dameTabla(t1, b)), c) $\land$ Pertenece?(Campos(dameTabla(t1, b)), c) $\land$ \\$\neg$hayJoin?(t1, t2, b)}
 {generarVistaJoin(cr, t, $b\_0$)}
 {Genera el Join, entre las tablas pasadas por parametro.}
 {O($(n+m)*(L+Log((n+m)))$)}
 {} 	

   \operacion{borrarJoin}{in t1: string, in t2: string, in b: base}{}
 {$b\_0$=b $\land$ hayJoin?(t1\, t2\, b)}
 {borrarJoin(t1\, t2\, $b\_0$)}
 {Borra el join entre tablas, pasadas por parametro.}
 {O($1$)}
 {}

%%%%%%% Otras Operaciones %%%%%%%%%%%

 \operacion{registros}{in t: string, in b: base}{res: conj(registro)}
 {t$\in$tablas(b)}
 {$res \igobs registros(t\, b)$}
 {Retorna el conjunto de registros correspondientes al nombre de tabla pasado por parametro}
 {O(1) por ref}
 {Se retorna el conjunto de registros por referencia.} 


 \operacion{vistaJoin}{in t1: string, in t2: string, in b: base}{res: conj(registro)}
 {\{t1\, t2\}$\subseteq$tablas(b) $\land$ hayJoin?(t1\,t2\,b)}
 {$res \igobs vistaJoin(t1\, t2\, b)$}
 {Retorna el conjunto de registros correspondientes al nombre de tabla pasado por parametro}
 {O($R*(L+ Log(n*m))$)}
 {Se retorna el conjunto de registros por referencia.} 

 \operacion{cantidadDeAccesos}{in t: string, in b: base}{res: nat}
 {t$\in$tablas(b)}
 {$res \igobs cantidadDeAccesos(t\,b)$}
 {Retorna la cantidad de modificaciones correspondientes al nombre de tabla pasado por parametro.}
 {O(1) por ref}
 {Se retorna res por referencia.} 

 \operacion{tablaMaxima}{in b: base}{res: string}
 {$\neg\emptyset$?(tablas(b))}
 {$res \igobs tablaMaxima(t\,b)$}
 {Retorna el nombre de la tabla con la mayor cantidad de modificaciones.}
 {O(1) por ref}
 {Se retorna el nombre de la tabla por referencia, no es modificable.} 
  
 \operacion{encontrarMaximo}{in t: string, in ct: conj(string), in b: base}{res: string}
 {\{t\}$\cup$ct $\subseteq$tablas(b)}
 {$res \igobs tablaMaxima(t\,b)$}
 {Retorna el nombre de alguna de las tablas con mayor cantidad de accesos.}
 {O($1$)}
 {Se retorna el nombre de la tabla por referencia, no es modificable.}
  
 \operacion{buscar}{in criterio: registro, in t: string, in b: base}{res: conj(ItConj(registro))}
 {t$\in$tablas(b)}
 {$res \igobs tablaMaxima(t\,b)$}
 {Retorna el conjunto de los iteradores de registro al conjunto de registros de la tabla.}
 {O($Cantidad de registros$)}
 {Se retorna por referencia, es modificable.}
  
 
\newpage
%%%%%%%%%%%%%%%%%%%%%%%%%%%%%%%%%%%%%%%%%%%%%%%%%%%%%%%%%%%%%%%%%%%%%%%%%%%%%
\subsection{Representación}

\serc{Base}
{
\donde{estr}{
\tupla{TablaMaxima: Tmax, Tablas: DiccTrie(NombreTabla; info\_tabla)}}
		\donde{info\_tabla}{\tupla{TActual: tabla, Joins: DiccTrie(NombreTabla; info\_join)}}
      	\donde{info\_join}{\tupla{
      								Rcambios: Cola(DatoCambio),
      								campoJ: campo, 
      								campoT: tipo,
      								JoinS: DiccTrie(string; itConj(registro)),
      								JoinN: DiccNat(nat; itConj(registro)),
      								JoinC: Conj(registro)}}
		\donde{Tmax}{\tupla{NomTabla: NombreTabla, $\#$Modif: Nat}}
		\donde{DatoCambio}{\tupla{Reg: Registro, NomOrigen: NombreTabla, Accion: Bool}}
}

\subsubsection*{Invariante de representación}

\begin{enumerate}
  \item El Nombre de las tablas es un String acotado.
  \item Índices es un arreglo de tamaño 2, que aloja el Índice correspondiente según el orden de creación.
  \item Para toda Dato que es clave en Índice, su significado llamémoslo sign está incluido en Registros.
  \item El nombre de la TablaMaxima está definido en Tablas.
  \item El nombre de la TablaMaxima corresponde a una del conjunto de tablas más modificadas.
  \item En Tmax $\#$Modif es igual a la cantidad de modificaciones de la tabla NomTabla.
  \item $\#$Modif de TablaMaxima es mayor o igual a todas las cantidades de modificaciones de las tablas en el conjunto de claves de Tablas.
  \item Cada NombreTabla definido en Tablas es igual al nombre de TActual en info$\_$tabla de su significado.
  \item Todas las claves de Joins, de todos los significados de las tablas definidas en Tablas, pertenecen al conjunto de claves de Tablas.
  \item Para todo info$\_$tabla, el nombre de TActual no está definido en Joins.
  \item para todo info$\_$tabla, el nombre de TActual pertenece al conjunto de claves del Joins de las claves de su Joins.
  \item Para todo info$\_$join, el tipo de campoJ es igual a campoT.
  \item Para todo DatoCambio, NomOrigen es igual a TActual o a la clave que le corresponde en Joins.
  \item Si campoT es true JoinS esta vacio, y si es false JoinN esta vacio.	
  \item En el caso campoT es false, en JoinS todas las claves, ese diccionario estan relacionadas cada una\\
  string S con solo iterador de conjunto a registro it, que apunta al registro r $\larr$Siguiente(it). Al cual si\\
  buscamos el dato d $\larr$Significado(r, campoJ), y hacemos ValorString(d) es igual a S
    
  \item En DatoCambio, si Accion es true entonces Reg pertence al conjunto de registros de NomOrigen.
  \item En DatoCambio, si Accion es false entonces Reg no pertence al conjunto de registros de NomOrigen.

\end{enumerate}


\subsubsection*{Función de abstracción}
\abs{Base}{BaseDeDatos}{b}{cb}

$ Tablas(b)\igobs tablas(cb)$ $\land$ \\
$((\paratodo{NombreTabla}{n}) Pertenece?(n, Tablas(b)) \impluego DameTabla(n, b) \igobs dameTabla(n,cb))$ \\
$((\paratodo{NombreTabla}{n1, n2}) (Definido?(n1, b.tablas) \land Definido?(n2, b.tablas)) \impluego HayJoin?(n, b) \igobs HayJoin?(n,cs) \yluego campoJoin(n1,n2, b)=campoJoin(n1,n2, cb))$ \\



%%%%%%%%%%%%%%%%%%%%%%%%%%%%%%%%%%%%%%%%%%%%%%%%%%%%%%%%%%%%%%%%%%%%%%%%%%%%%%%%%%
\newpage

\subsection{Algoritmos}

\algoritmo{tablas}{in b: estr}{res: ConjTrie(string)}{
  \State $res \larr DiccClaves(b.tablas) $ \complejidad{O(1) por ref}
}{O(1) por ref}

\algoritmo{dameTabla}{in t: string, in b: estr}{res: tabla}{
  \State $info \larr Significado(b.tablas, t) $ \complejidad{O(1) por ref}
  \State $res \larr info.TActual $ \complejidad{O(1) por ref}
}{O(1) por ref}

\algoritmo{hayJoin?}{in t1: string, in t2: string, in b: estr}{res: bool}{
	\State   res $\larr$(Definido?(Obtener(b, t1).Joins,t2) $\larr$ Definido?(Obtener(b, t2).Joins,t1))  \complejidad{O(1) por ref}
}{O(1) por ref}

\algoritmo{campoJoin}{in t1: string, in t2: string, in b: estr}{res: campo}{
		\State res$\larr$ Obtener(Obtener(b, t1).Joins, t2).campoJ \complejidad{O(1) por ref}
}{O(1) por ref}

%%%%%%%%%%%%% generadores %%%%%%%%%%%%%
\algoritmo{nuevaDB}{}{res: estr)}{ 
	\State String s \complejidad{O(1) }
	\State Nat n $\larr$ 0\complejidad{O(1)}
	\State TablaMaxima $\larr$ $< s, 0 >$\complejidad{O(1) por ref}
  	\State res $\larr$ $\langle TablaMaxima, vacio()\rangle$  \complejidad{O(1) por ref} 
}{O(1) por ref}

\algoritmo{agregarTabla}{in t: tabla, in/out b: estr}{}{
	\State	info\_tabla $\larr$ $\langle cantidadDeAccesos(t), t, vacio()\rangle$  \complejidad{O(1) por ref}
	\State	Definir(b.tablas, nombre(t), info\_tabla)								 \complejidad{O(1) por ref}
}{O(1) por ref}

\algoritmo{insertarEntrada}{in reg: registro, in t: string, in/out b: estr}{}{
	\State	Obtenemos la tabla es O(1) por ref porque su nombre esta acotado.
	\State	info\_tabla infoT $\larr$	Obtener(b.tablas, t).TActual		\complejidad{O(1) por referencia}
	\State	Agrego el registro a la tabla. \complejidad{O(1) por ref}
	\State	Tabla T $\larr$ infoT.TActual \complejidad{O(1) por ref}
	\State	agregarRegistro(reg, T)								\complejidad{O(L+Log(n))}
	\State	Ahora si hay Joins actualizo la informacion temporal de cada Join.
	\If{$\neg$Vacio?(infoT.Joins)} \complejidad{O(1) por ref}
				\State	ItConjString(String) itNomTab $\larr$ CrearIt(Claves(infoT.Joins))\complejidad{O(1) por ref}
				\While{HaySiguiente?(itNomTab)}								\complejidad{O(W)}
					\State	info\_join infoJ $\larr$ Obtener(infoT.Joins, Siguiente(NomTab)) \complejidad{O(1) por ref}
					\State	Encolar(infoJ.Rcambios, $\langle reg, Siguiente(NomTab),  true\rangle$) \complejidad{O(1) por ref}
					\State	Encolar es O(L) porque se copia un registro con su cantidad de campos acotada
					\State	y los valores string copiarlos tiene costo O(L), siendo L el valor string mas largo.
					\State	Avanzar(itClaves)\complejidad{O(1) por ref}
				\EndWhile
	\EndIf
	\If{CantidadDeAccesos(T) $\>$b.TablaMaxima.\#Modif}
			\State	b.TablaMaxima.NomTabla $\larr$ Copiar(Nombre(T)) \complejidad{O(L)}
			\State	b.TablaMaxima.\#Modif $\larr$ Copiar(CantidadDeAccesos(T))\complejidad{O(1) por ref}
	\EndIf
	\State W la cant de tablas en la base
}{O(W) }

\algoritmo{borrar}{in cr: registro, in t: string, in/out b: estr}{}{
	\State info\_tabla infoT $\larr$	Obtener(b.tablas, t).TActual		\complejidad{O(1) por ref}
	\State	 Tabla T $\larr$ infoT.TActual \complejidad{O(1) por ref}
	\State NombreTabla NomTab $\larr$ NombreTabla(T)
	\State La eliminacion en primera etapa depende de si hay
	\State	 joins con la tabla pasada por parametro
	\If{$\neg\emptyset$?(Claves(infoT.Joins))} \complejidad{O(1) por ref}
			\State Creo el iterador, para navegar los nombres de tablas con los que tiene Join
			\State itNom $\larr$ CrearIt(Claves(infoT.Joins)) \complejidad{O(1) por ref}
			\While{HaySiguiente?(itNom)} \complejidad{O(Cant de tablas)}
					\State info\_join infoJ $\larr$ Siguiente(itNom) \complejidad{O(1) por ref}
					\State Verifico si el Join esta creado en base al 
					\State campo del cr pasado por parametro.
					\If{infoJ.campoJ=DameUno(Campos(cr))} \complejidad{O(1) por ref}
							\State	Entonces solo actualizo la cola temporal del join
							\State	Registro reg $\larr$ Buscar(cr, T) \complejidad{O(L + Log(n))}
							\State	Encolar(infoJ.Rcambios, $\langle reg, Siguiente(itNom),  False\rangle$)\complejidad{O(L)}
					\Else
							\State	Si el campo del criterio de borrado, es distinto que el campo del Join.
							\State	Primero busco que registros coinciden con el criterio
							\State	en el peor caso con complejidad O(cantidad de registros de t) sin indices.
							\State	Conj(Registro) cjc $\larr$ Buscar(cr, T)\complejidad{O(L + Log(n))}
																						% Coincidencias(cr, Registros(r))
							\State	ItConj(Registro) itReg $\larr$ CrearIt(cjc)
							\State	Luego agrego estos registros a la cola temporal de cambios del join
							\While{HaySiguiente?(itReg)}
									\State	Encolar(infoJ.Rcambios, $\langle Siguiente(itReg), NomTab,  False\rangle$) \complejidad{O(L)}
									\State Avanzar(itReg)\complejidad{O(1) por ref}
							\EndWhile
					\EndIf
			\EndWhile
	\EndIf
	\State	Habiendo actualizado las colas temporales de los joins con los 
	\State registros que cumplen con el criterio de borrado
	\State de los joins correspondientes.
	\State	Elimino del conjunto de registros, aquellos que cumplen el criterio de borrado.
	\State	Siendo n la cantidad de registros de t y T la cantidad de tablas en la base
	\State	En peor caso con costo O(n)
	\State	borrarRegistro(r, $T\_$actual)						\complejidad{O(T*L + n)}
	\If{CantidadDeAccesos(T) $\>$b.TablaMaxima.\#Modif}
			\State	b.TablaMaxima.NomTabla $\larr$ Copiar(Nombre(T)) \complejidad{O(L)}
			\State	b.TablaMaxima.\#Modif $\larr$ Copiar(CantidadDeAccesos(T))\complejidad{O(1) por ref}
	\EndIf
}{O(T + Log(n))}

\algoritmo{generarVistaJoin}{in t1: string, in t2: string, in c: campo, in/out b: estr}{}{
	\State	Cola(DatoCambio) Rcambio $vacia()$
	\State	Campo campoJ $\larr$
	\State	Tipo campoT $\larr$ tipoCampo(c, Ta1)
	\State	DiccTrie(string; itConj(registro))  JoinS $\larr$ Vacio()
	\State	DiccNat(nat; itConj(registro)) JoinN  $\larr$ Vacio()
	\State	Conj(Registro) JoinC $\larr$ vacio() \complejidad{O(1) por ref}
      								
	\State	Tabla Ta1 $\larr$	Obtener(b.tablas, t1).Tactual		\complejidad{O(1) por ref}
	\State	Tabla Ta2 $\larr$	Obtener(b.tablas, t2).Tactual		\complejidad{O(1) por ref}
	\State	Nat n $\larr$Cardinal(Registros(Ta1))\complejidad{O(1) por ref}
	\State	Nat m $\larr$Cardinal(Registros(Ta2))\complejidad{O(1) por ref}
	\State	Conj(dato) cjd1$\larr$dameColumna(c, Registros(Ta1)) \complejidad{O(nLog(n))}
	\State	Conj(dato) cjd2$\larr$dameColumna(c, Registros(Ta2)) \complejidad{O(mLog(m))}
	\State	Ahora busco la interseccion de estos datos entre los conjuntos cjd1 y cjd2
	\State	ItConj(dato) itcjd1 $\larr$ CrearIt(cjd1) \complejidad{O(1) por ref}
	\State	DiccString(String, bool) ds $\larr$ vacio() \complejidad{O(1) por ref}
	\State	DiccNat(Nat, bool) dn $\larr$ vacio()\complejidad{O(1) por ref}
	\State	Conj(Dato) cjD $\larr$ vacio()
	\State	Cargo los valores de t1 en un dicc que depende del tipo del campo c
	\If{$\neg$campoT} \complejidad{O(1) por ref}
		\State El tipo del campo es string
		\While{HaySiguiente?(itcjd1)} \complejidad{O(n)}
			\State Sabemos que el dameColumna no tiene valores repetidos
	     	\State Definir(ds, ValorString(Siguiente(itcjd1)), true)\complejidad{O(1) por ref}
			\State Avanzar(itcjd1)\complejidad{O(1) por ref}
		\EndWhile	
		\State	ItConj(dato) itcjd2 $\larr$ CrearIt(cjd2) \complejidad{O(1) por ref}	
		\State	Ahora buscamos la interseccion
		\While{HaySiguiente?(itcjd2)} \complejidad{O(m)}
			\State Sabemos que el dameColumna no tiene valores repetidos
			\State Bool Def $\larr$ Definido?(ds, ValorString(Siguiente(itcjd2)))\complejidad{O(1) por ref}
			\If{Def}
				AgregarRapido(cjD, Siguiente(itcjd2))\complejidad{O(1) por ref}
		     \EndIf
			\State	Avanzar(itcjd1)\complejidad{O(1) por ref}
		\EndWhile	
	\Else
		\State El tipo del campo es nat
		\While{HaySiguiente?(itcjd1)} \complejidad{O(n*log(n))}
			\State Sabemos que el dameColumna no tiene valores repetidos
	     	\State Definir(dn, ValorNat(Siguiente(itcjd1)), true)\complejidad{O(Log(n))}
			\State Avanzar(itcjd1)\complejidad{O(1) por ref}
		\EndWhile	
		\State	ItConj(dato) itcjd2 $\larr$ CrearIt(cjd2) \complejidad{O(1) por ref}	
		\State	Ahora buscamos la interseccion
		\While{HaySiguiente?(itcjd2)} \complejidad{O(m*Log(n))}
			\State Sabemos que el dameColumna no tiene valores repetidos
			\State Bool Def $\larr$ Definido?(dn, ValorString(Siguiente(itcjd2)))\complejidad{O(log(n))}
			\If{Def}
				AgregarRapido(cjD,Siguiente(itcjd2))\complejidad{O(1) por ref}
		     \EndIf
			\State	Avanzar(itcjd2)\complejidad{O(1) por ref}
		\EndWhile
	\EndIf
	\State Si hay algun valor en la interseccion, con esos valores hago el join.
	\If{$\neg$Vacio?(cjD)}\complejidad{O(1) por ref}
		\State Caso Strings
		\State its $\larr$ CrearIt(cjD) \complejidad{O(1) por ref}
		\While{HaySiguiente?(its)} \complejidad{O((n+m)*(L+Log(n*m)))}
			\State	Registro regModelo $\larr$ Definir(vacio(), c, Siguiente(its))
			\State	Como el campo es clave en ambas tablas
			\State	Si en ambas tablas hay indice en base a c y la complejidad es 
			\State	O(L+Log(n*m)) sino es O(cardinal de registros de la tabla)
			\State	Asumimos que es el mejor caso para la complejidad.
			\State Registro r1 Siguiente(cj1) $\larr$ BuscarEnTabla(regModelo, ta1)\complejidad{O(L+Log(n*m))}
			\State Registro r2 Siguiente(cj1) $\larr$ BuscarEnTabla(regModelo, ta1)\complejidad{O(L+Log(n*m))}
			\If{$\neg$campoT}
				\State itConj(Registro) itnuevo AgregarRapido(JoinC, UnirRegistros(r1, r2))\complejidad{O(1) por ref}
				\State Definir(JoinS, ValorString(Siguiente(its)), itnuevo)\complejidad{O(1) por ref}
			\Else
				\State itConj(Registro) itnuevo AgregarRapido(JoinC, UnirRegistros(r1, r2))\complejidad{O(1) por ref}
				\State Definir(JoinN, ValorNat(Siguiente(its), itnuevo)\complejidad{O(Log(n+m))}
			\EndIf
			\State Avanzar(its)\complejidad{O(1) por ref}
		\EndWhile
	\EndIf
	\State info\_join $\larr$ $<Rcambios, campoJ, campoT, JoinS, JoinN, JoinC>$\complejidad{O(1) por ref}
	\State Definir(Ta1.Joins, t2, info\_join)\complejidad{O(1) por ref}
	\State Definir(Ta2.Joins, t1, info\_join)\complejidad{O(1) por ref}
}{O(1) por ref}
%%%%%%%%%%%%%%%%%%%%
%%%%%%%%%%%%%%%%%%
%%%%%%%%%%%%%
%%%%%%%%%
%	% Vemos si existe un indice en ambas tablas para c
%	\If{Pertenece?(Indices(T\_actual1), c) $\land$ Pertenece?(Indices(T\_actual1), c)} \complejidad{O(1) por ref}
%			\State	ind1 $\larr$ Obtener(T\_actual1.Indices, c)  \complejidad{O(calcular)}
%			\If{tipoCampo(T\_actual1, c)} \complejidad{O(1) por ref}
%					\State	ConjNat(Nat) cjNat $\larr$ vacio() \complejidad{O(1) por ref}
%					\State	itvalores $\larr$ CrearIt(ind1.PorNat.DiccClaves) 
%					\While{HaySiguiente(itvalores)}
%							\State	Registro r $\larr$ Obtener(ind1.PorNat, Siguiente(itvalores))
%							\State	Nat n $\larr$ ValorNat(Obtener(r, c))
%							\If{$\neg$ Pertenece?(cjNat, n)}
%									\State	AgregarRapido(cjNat, r)
%							\EndIf
%							\State	Avanzar(itvalores)
%					\EndWhile
%					\State	ind2 $\larr$ Obtener(T\_actual2.Indices, c)		
%					\State	itvalores $\larr$ CrearItConjNat(ind2.PorNat.DiccClaves)
%					\While{HaySiguiente(itvalores)}
%							\State	Registro r $\larr$ Obtener(ind2.PorNat, Siguiente(itvalores))
%							\State	Nat n $\larr$ ValorNat(Obtener(r, c))
%							\If{$\neg$ Pertenece?(cjNat, n)}
%									\State	AgregarRapido(cjNat, r)
%							\EndIf
%							\State	Avanzar(itvalores)
%					\EndWhile
%%					\While{HaySiguiente(itvalores)}
%%							\State	r1 $\larr$ Obtener(ind1.PorNat, Siguiente(itvalores))
%%							\State	r2	$\larr$ Obtener(ind2.PorNat, Siguiente(itvalores))
%%							\State	cj1 $\larr$ AgregarRapido(vacio(), r1)
%%							\State	cj2 $\larr$ AgregarRapido(vacio(), r2)
%%							\State	nuevor $\larr$ combinarRegistros(c, cj1, cj2)
%%							\State	AgregarRapido(Join,DameUno(nuevor))
%%							\State	Avanzar(itvalores)
%%					\EndWhile
%			\Else
%					\State	itvalores $\larr$ CrearItConjString(ind1.PorString.DiccClaves)
%					\While{HaySiguiente(itvalores)}
%							\State	r1 $\larr$ Obtener(ind1.PorString, Siguiente(itvalores))
%							\State	r2	$\larr$ Obtener(ind2.PorString, Siguiente(itvalores))
%							\State	cj1 $\larr$ AgregarRapido(vacio(), r1)
%							\State	cj2 $\larr$ AgregarRapido(vacio(), r2)
%							\State	nuevor $\larr$ combinarRegistros(c, cj1, cj2)
%							\State	AgregarRapido(Join, DameUno(nuevor))
%							\State	Avanzar(itvalores)
%					\EndWhile
%			\EndIf
%	\Else
%			\State	cjr1 $\larr$ T\_actual1.registros
%			\State	cjr2 $\larr$ T\_actual1.registros
%			\State	Join$\larr$combinarRegistros(c, cjr1,cjr2)
%	\EndIf
%	\State info\_join $\larr$ $\langle 0, vacio(), vacio(), c, tipoCampo(T\_actual1, c), Join\rangle$
%	\State Definir(b.Joins,	$\langle t1, t2\rangle$ , info\_join)


\algoritmo{borrarJoin}{in t1: string, in t2: string, in/out b: estr}{}{
	\State Tabla tab1 $\larr$ DameTabla(t1, b)
	\State Tabla tab2 $\larr$ DameTabla(t2, b)
	\State Borrar(tab1.Joins, t2)
	\State Borrar(tab2.Joins, t1)
}{O(1) por ref}


% \operacion{buscar}{in criterio: registro, in t: string, in b: base}{res: conj(registro)}
% {t$\in$tablas(b)}
% {$res \igobs tablaMaxima(t\,b)$}
% {Retorna ...}
% {O($calcular$)}
% {Se retorna el nombre de la tabla por referencia.}
\algoritmo{buscar}{in criterio: registro, in t: string, in b: base}{res: conj(ItConj(registro))}{
	\State	Busco los datos de la tabla.
	\State	info\_tabla infot $\larr$ Significado(b.tablas, t) \complejidad{O(1) por ref}
	\State	Tabla tab $\larr$ infot.TActual \complejidad{O(1) por ref}
	\State	res $\larr$ BuscarEnTabla(criterio, tab) 
}{O(1) por ref}





\algoritmo{registros}{in t: string, in b: base}{res: Conj(registros)}{
  \State $info \larr Significado(b.tablas, t) $ \complejidad{O(1) por ref}
  \State $tab \larr info.TActual $ \complejidad{O(1) por ref}
  \State $res \larr registros(tab) $ \complejidad{O(1) por ref}
}{O(1) por ref}


\algoritmo{vistaJoin}{in t1: string, in t2: string, in/out b: estr}{}{
	\State	info$\_$tabla	infot1 $\larr$ Significado(b.Tablas, t1)\complejidad{O(1) por ref}
	\State	Tabla	tab1 $\larr$ infot1.TActual	\complejidad{O(1) por ref}
	\State	info$\_$tabla	infot2 $\larr$ Significado(b.Tablas, t1)\complejidad{O(1) por ref}
	\State	Tabla	tab2 $\larr$ infot2.TActual	\complejidad{O(1) por ref}
	\State	info$\_$join	infoj $\larr$ Significado(infot1.Joins, t2) \complejidad{O(1) por ref}
	\State	Campo c $\larr$ infoj.campoJ \complejidad{O(1) por ref}
	\If{$\neg$EsVacia?(infoj.Rcambios)} \complejidad{O(1) por ref}
		\State	Hubo cambios desde la generacion del join o del ultimo vistaJoin
		\While{$\neg$EsVacia?(infoj.Rcambios)}
				\State	DatoCambio data $\larr$ Proximo(infoj.Rcambios) \complejidad{O(1) por ref}
				\State	Desencolar(infoj.Rcambios) \complejidad{O(1) por ref}
				\State	Registro r $\larr$ data.Reg \complejidad{O(1) por ref}
				\If{data.Accion}
					\State 	La accion es agregar un registro al join
					\State 	Para hacer esto necesito saber a que tabla se agrego el registro
					\State 	NombreTabla NomTorigen $\larr$ data.NomOrigen \complejidad{O(1) por ref}
					\State 	Sabiendo la tabla de origen, necesito identificar la otra tabla para ver si
					\State 	hay un registro con el mismo valor para el campo del join.
					\State	Armo un registro auxiliar regModelo con el campo
					\State	Dicc(campo,dato) regModelo $\larr$ vacio()
					\State	Definir(regModelo, c, Significado(r, c))
					\State	Si hay indice en la tabla para el campo c y ademas es campo clave
					\State	BuscarEnTabla(tab, regModelo) tiene complejidad O(L) u O(Log(n))
					\State	dependiendo del tipo del campo c.
					\State	Si no hay indice para el campo c entonces BuscarEnTabla(tab, regModelo)
					\State	tiene complejidad O(m) siendo m la cant de registros en tab.
					\State 	Registro rotro $\larr$ vacio()
					\If{NomTorigen=t1}
						\State	Entonces el otro registro lo tengo que buscar en t2
						\State	Registro rotro $\larr$ Siguiente(BuscarEnTabla(tab2, regModelo))
					\Else
						\State	Entonces el otro registro lo tengo que buscar en t1
						\State	Registro rotro $\larr$ Siguiente(BuscarEnTabla(tab1, regModelo))
					\EndIf
					\If{$\neg$Vacio?(rotro)}
						\State Si habia un registro que cumplia con lo pedido
						\State	Registro rnuevo $\larr$ UnirRegistros(r, rotro)
						\If{info$\_$join.campoT}
							\State	El tipo del campo es Natural.
							\State	itConj(registro) itnew $\larr$ AgregarRapido(info$\_$join.JoinC, rnuevo)
							\State	Definir(info$\_$join.JoinN, key, itnew)
						\Else
							\State	El tipo del campo es String.
							\State	itConj(registro) itnew $\larr$ AgregarRapido(info$\_$join.JoinC, rnuevo)
							\State	Definir(info$\_$join.JoinS, key, itnew)
						\EndIf
					\EndIf
				\Else	CasoBorrado
					\If{info$\_$join.campoT}
						\State	El tipo del campo es Natural.
						\State	itConj(registro) itcjr $\larr$ Significado(info$\_$join.JoinN, key)
						\State	Borrar(info$\_$join.JoinN, key)
						\State	EliminarSiguiente(itcjr)
					\Else
						\State	El tipo del campo es String.
						\State	itConj(registro) itcjr $\larr$ Significado(info$\_$join.JoinN, key)
						\State	Borrar(info$\_$join.JoinS, key)
						\State	EliminarSiguiente(itcjr)
					\EndIf
				\EndIf
		\EndWhile
	\EndIf
	\State res $\larr$ info$\_$join.JoinC
}{O(Tamaño(infoj.Rcambios)*)}

\algoritmo{cantidadDeAccesos}{in t: string, in b: base}{res: nat}{
  \State $info \larr Significado(b.tablas, t) $ \complejidad{O(1) por ref}
  \State $tab \larr info.TActual $ \complejidad{O(1) por ref}
  \State $res \larr cantidadDeAcccesos(tab) $ \complejidad{O(1) por ref}
}{O(1) por ref}

\algoritmo{tablaMaxima}{in b: base}{res: string}{
  \State $res \larr b.TablaMaxima.NomTabla $ \complejidad{O(1) por ref}  
}{O(1) por ref}
