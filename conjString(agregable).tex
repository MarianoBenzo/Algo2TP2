\section{Modulo Conjunto Lexicografico}

\subsection{Interfaz}

\sexc{Conjunto(String)}
$\textbf{usa}$ Bool, Nat, String
\generos{conjString}

\subsubsection*{Operaciones básicas}

\operacion{Vacio}{}{res : diccString(String, \sigma)}
 {$true$}
 {$res \igobs vacio()$}
 {Crea un nuevo conjunto}
 {$O(1)$}
 {}

 
 \operacion{Pertenece?}{in d:conjString in n:String}{res:Bool}
  {$true$}
  {$res \igobs def?(n,d)$}
  {Indica si el elemento pertenece al conjunto}
  {$O(Longitud(n))$}
  {}
  
   \operacion{Cardinal}{in d:conjString}{res:Nat}
  {$true$}
  {$res \igobs \# d$}
  {Indica el cardinal del conjunto}
  {$O(Longitud(n))$}
  {}

 \operacion{Agregar}{in/out d:conjString in n:String}{}
  {$d=d_0$}
  {$d \igobs Ag(n, d_0)$}
  {Agrega el elemento n al conjunto}
  {$O(Longitud(n))$}
  {$s$ se define por copia}

 \operacion{Borrar}{in/out d:conjString in n:String}{}
 {$d \igobs d_0 \land n \in d$}
 {$d \igobs d_0 - \{ n \}$}
 {Elimina el elemento n}
 {$O(Longitud(n))$}
 {}


 
  \Titulo{Operaciones del iterador}

  El iterador que presentamos permite modificar el diccionario recorrido.
  

  Para simplificar la notacion, vamos a utilizar clave y significado en lugar de $\Pi_1$ y $\Pi_2$ cuando utilicemos una tupla($String,\sigma$).
 \operacion{CrearIt}{in d: conjString in n:String}{res: itDiccString($String,\sigma$)}
 {$true$}
 {alias(esPermutacion(SecuSuby($res$), $d$)) $\land$ vacia?(Anteriores($res$))}
 {crea un iterador bidireccional del diccionario, que apunta al primer elemento del mismo en orden lexicografico.}
  {$O(CL*long(k))$ Donde $CL$ es la cantidad de claves de d y $k$ la palabra mas larga de d}
  {hay aliasing entre los significados en el iterador y los del diccionario}


  \operacion{HaySiguiente}{in it:itConjString in n:String}{res:bool}
	{$true$}  
  {$res$ $\igobs$ haySiguiente?($it$)}
  {devuelve \texttt{true} si y solo si en el iterador todavia quedan elementos para avanzar.}
  {$O(1)$}
  {}

  \operacion{HayAnterior}{in it: itConjString in n:String}{res:bool}
	{$true$}    
  {$res$ $\igobs$ hayAnterior?($it$)}
  {devuelve \texttt{true} si y solo si en el iterador todavia quedan elementos para retroceder.}
  {$O(1)$}
  {}

  \operacion{Siguiente}{in it: itConjString in n:String}{res:tupla(String, \sigma)}
  {HaySiguiente?($it$)}
  {alias($res$ $\igobs$ Siguiente($it$))}
  {devuelve el elemento siguiente del iterador.}
  {$O(1)$}
  {$res$.significado es modificable si y solo si $it$ es modificable.  En cambio, $res$.clave no es modificable.}
  

  \operacion{Anterior}{in it: itConjString in n:String}{tupla(clave: $String$, significado: $\sigma$)}
  {HayAnterior?($it$)}
  {alias($res$ $\igobs$ Anterior($it$))}
  {devuelve el elemento anterior del iterador.}
  {$O(1)$}
  {$res$.significado es modificable si y solo si $it$ es modificable.  En cambio, $res$.clave no es modificable.}




  \operacion{Avanzar}{inout it: itConjString in n:String}{}
  {$it = it_0$ $\land$ HaySiguiente?($it$)}
  {$it$ $\igobs$ Avanzar($it_0$)}
  {avanza a la posicion siguiente del iterador.}
  {$O(1)$}
  {}

  \operacion{Retroceder}{inout it: itConjString in n:String}{}
  {$it = it_0$ $\land$ HayAnterior?($it$)}
  {$it$ $\igobs$ Retroceder($it_0$)}
  {retrocede a la posicion anterior del iterador.}
  {$O(1)$}
  {}



\subsection{Representacion}
diccString\serc{cLex)}{
	\\
	\donde{cLex}{\tupla{dicL: diccString($String$, bool), cardinal: Nat}}
}





\subsubsection*{Respresentación del iterador}
  \Titulo{Representacion del iterador}

  El iterador del conjunto lo recorre en orden lexicográfico. Los significados están por referencia.
itConjString\serc{itcLex)}{
	\\
	\donde{itcLex}{itDiccString($String$, Bool)}
}


  

  ~

  %\tadOperacion{Join}{secu($\alpha$)/a,secu($\beta$)/b}{secu(tupla($\alpha,\beta$))}{long($a$) $=$ long($b$)}
  %\tadAxioma{Join($a$, $b$)}{\IF vacia?($a$) THEN \secuencia{} ELSE \secuencia{$\langle${prim($a$), prim($b$)}$\rangle$}[Join(Fin($a$), Fin($b$))] FI}



\subsection{Algoritmos}

\algoritmo{iVacio}{}{res:diccString}{
  \State $res.dicL \larr Vacio()$ \complejidad{O(1)}
  \State $res.cardinal \larr 0$ \complejidad{O(1)}
}{O(1)}
