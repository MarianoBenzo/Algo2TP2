
%\section{Pos es tupla(x:Nat, y:Nat)}

%\section{Placa es Nat}

%\section{Nombre es String}

\section{Registro}

\subsection{Interfaz}

\sexc{registro}
$\textbf{usa}$  
\generos{dicc(campo,dato), itDicc(campo,dato)}




\subsubsection*{Operaciones}
%%%%%% observadores basicos %%%%%%




%%%%%%%%%%%%%%%%%%%%%%% Otras Operaciones %%%%%%%%%%%%%%%%%%%%%%%%%%%%%%%%%%

\operacion{campos}{in r: dicc(campo, dato)}{res : itConj(campo)}
 {true}
 {$res \igobs claves(r)$}
 {Devuelve un conjunto de campos que son claves del registro ingresado por parametro}
 {O($calc$) [$\Theta(\sum_{c' in  c}equal(c,c'))$, donde $c$ $=$ claves($d$)]}
 {Se devuelve un iterador al conjunto, hay aliasing}

\operacion{borrar?}{in crit: dicc(campo, dato), in r: dicc(campo,dato)}{res : bool}
 {#campos(crit) = 1}
 {$res \igobs borrar?(crit, r)$}
 {Devuelve true si y solo si todos los campos de crit pertenecen a campos de r}
 {O($calc$)} % O(obtener(#cc))
 {}

\operacion{agCampos}{in/out  r1: dicc(campo, dato) in  r2: dicc(campo, dato)}{res : reg}
 {$ r1 \igobs r1_0 $}
 {$r1 \igobs agregarCampos(r1_0, r2) $}
 {agrega los datos de los campos faltantes de r2 a los campos de r1}
 {O($calcular$)  }
 {}
 
\operacion{copiarCampos}{in/out  r_1: dicc(campo, dato), in  cc: conj(campo), in r2: reg}{}
 {$r1 \igobs r1_0 \and  cc in  campos(r2) $}
 {$r1 \igobs copiarCampos(cc, r1_0, r2)$}
 {copia los datos que se encuentran en los campos de r2 a r1}
 {O($calc$)}
 {}

\operacion{coincideAlguno}{in  r1: dicc(campo, dato), in cc: conj(campo), in  r2: dicc(campo, dato)}{res : bool}
 {cc \subseteq campos(r1)\cap campos(r2)}
 {$res \igobs coincideAlguno(r1, r2)$}
 {Devuelve true si y solo si alguno de los campos(dato) de cc pertenece a r1 y r2}
 {O($calc$)}
 {}
 
\operacion{coincidenTodos}{in r1: dicc(campo, dato), in cc: conj(campo), in r2: dicc(campo, dato)}{res : bool }
 {cc \subseteq campos(r1)\cap campos(r2)}
 {$res \igobs $}
 {Devuelve ture si y solo si todos los campos(dato) de cc pertenecen a r1 y r2}
 {O($calc$) } 
 {}
 
\operacion{enTodos}{in c: campo, in cr: conj(registro)}{res : bool}
 {true}
 {$res \igobs enTodos(c,cr)$}
 {Devuelve true si y solo si campo c pertenece a los campos de cada uno de los registros cr}
 {O($calc$)}
 {}

\operacion{combinarTodos}{in c: campo, in r1:dicc(campo, dato), in cr: conj(dicc(campo,dato))}{res : conj(itDicc(campo, dato))}
 {c $in $ campos(r1) $\land$ enTodos(c,cr)}
 {$res \igobs cantidadDeAccesos(t)$}
 {Devuelve }
 {O($1$)}
 {}

%%%%%%%%%%%%%%%%%%%%%%%%%%%%%%%%%%%%%%%%%%%%%%%%%%%%%%%%%%%%%%%%%%%%%%%%%%%%%
\subsection{Representación}

\serc{Registro}{
  \donde{dicc(campo, dato)}
      \donde{dic}{diccTrie(Dato,campo)}
  }
}

\subsubsection*{Invariante de representación}

\begin{enumerate}
  \item El Nombre de la tabla es un String acotado.
  \item Indices es un arreglo de tamaño 2, que aloja el Indice correspondiente segun el orden de creacion.
  \item Para toda Dato que es clave en Indice, su significado llamemoslo sign esta incluido en Registros.
  \item 

\end{enumerate}

\subsubsection*{Función de abstracción}
\abs{sistema}{CampusSeguro}{s}{cs}
$s.campus \igobs campus(cs)$ $\land$ \\
$s.estudiantes \igobs estudiantes(cs)$ $\land$ \\
$s.hippies \igobs hippies(cs)$ $\land$ \\
$s.agentes \igobs agentes(cs)$ $\land$ \\
$((\paratodo{nombre}{n}) s.hippies.definido(n) \impluego s.hippies.obtener(n) \igobs posEstYHippie(n,cs)$ $\lor$ \\
$(\paratodo{nombre}{n}) s.estudiantes.definido(n) \impluego s.estudiantes.obtener(n) \igobs posEstYHippie(n,cs))$ \\
$(\paratodo{placa}{pl}) s.agentes.definido(pl) \impluego s.estudiantes.obtener(pl).pos \igobs posAgente(pl,cs))$ \\
$(\paratodo{placa}{pl}) s.agentes.definido(pl) \impluego s.estudiantes.obtener(pl).cantSanciones \igobs cantSanciones(pl,cs))$ \\
$(\paratodo{placa}{pl}) s.agentes.definido(pl) \impluego s.estudiantes.obtener(pl).cantCapturas \igobs cantCapturas(pl,cs))$ \\

%%%%%%%%%%%%%%%%%%%%%%%%%%%%%%%%%%%%%%%%%%%%%%%%%%%%%%%%%%%%%%%%%%%%%%%%%%%%%%%%%%
\newpage

\subsection{Algoritmos}

  \algoritmo{campos}{in r: dicc(campo, dato)}{conj(campo)}{
  \State $res \larr CrearItConjTrie(r.ClavesDicc)$ \complejidad{O(1)}
}{O(1)}
 
   \algoritmo{borrar?}{in cr: dicc(campo, dato) in r: dicc(campo, dato)}{res:bool}{
    \State  res $\larr$ false;
    \State  \While{HaySiguiente(cr)}
        \State  res $\larr$   coincideAlguno(r,cc,Siguiente(cr)) $\lor$res;
        \State  Avanzar(cr);
  \State  \EndWhile   
 }{O(calcular)}

 \algoritmo{agregarCampos}{in r1: dicc(campo, dato) in r2: reg}{res:reg}{
    \State  res $\larr$ CrearItConj(vacio());
    \State  \While
        \State  \If 
                \State  
 

 
  \algoritmo{copiarCampos}{in cc: conj(campo) in/out r1: dicc(campo, dato) in r2: reg}{}{
    \State  res $\larr$ min( dameColumna(  c,  CrearItConj(t.registros))  );

 
   \algoritmo{coincideAlguno}{in r1: dicc(campo, dato) in cc: conj(campo), in r2: reg}{res:bool}{
    \State  res $\larr$ 

 
    \algoritmo{coincidenTodos}{in r1: dicc(campo, dato) in cc: conj(campo), in r2: reg}{res:bool}{
    \State  res $\larr$ 
    \State  \While{
        \State  \If 
                \State 
        \State  \EndIf        
  \State  Avanzar(cr);
  \State  \EndWhile       
 }{O(calcular)}
 
    \algoritmo{enTodos}{in c:campo, in cr: conj(registro)}{res:bool}{
    \State  res $\larr$ True;
    \State  ; 
    \State  \While
        \State  
        \State 
  \State  \EndWhile       

  \algoritmo{combinarTodos}{in c: campo, in r1: dicc(campo, dato) in cr: conj(registro)}{res:itConj(reg)}{
      \If {$\neg(encerrado?(campus,campus.hippies.obtener(nombre)))$} \complejidad{O(long(nombre))}
        \State 
        \State \complejidad{O(long(nombre))}
      \EndIf
  }{O(long(nombre) + $N_e$)}

 