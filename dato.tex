
%\section{Pos es tupla(x:Nat, y:Nat)}

\section{Tipo es Bool}

%\section{Nombre es String}

\section{Dato($\alpha$)}

\subsection{Interfaz}

\sexc{dato}
$\textbf{usa}$  
\generos{nat, string, tipo}

%%%%%%%%%%%%%%%%%%%%%%%%%%%%%%%%%%%%%%%%%%%%%%%%%%%%%%%%%%%%%%%%%%%%%%%%%%%%

\subsubsection*{Operaciones}


%%%%%% observadores basicos %%%%%%

\operacion{tipo?}{in d: dato }{res : tipo}
 {true}
 {$res \igobs tipo?(d)$}
 {Devuelve el tipo del dato ingresado por parametro.}
 {O($1$)}
 {Se retorna res por referencia.}

\operacion{valorNat}{in d: dato }{res : nat}
 {Nat?(d)}
 {$res \igobs valorNat(t)$}
 {Devuelve valor numerido del dato por parametro.}
 {O($1$)}
 {Se devuelve res por referencia.}

\operacion{valorString}{in d: dato }{res : string}
 {String?(d)}
 {$res \igobs valorString(t)$}
 {Devuelve valor del dato por parametro.}
 {O($1$)}
 {Se devuelve res por referencia.}
 
 %%%%%% generadores %%%%%%%

 \operacion{datoNat}{in n: $\alpha$,in tipoDelDato: tipo}{res : dato}
 {$tipoDelDato$}
 {$res \igobs datoNat(n,tipoDelDato)$}
 {Crea un dato de valor numerico.}
 {O($1$)}
 {Se devuelve res por referencia.}
 
 \operacion{datoStr}{in n: $\alpha$,in tipoDelDato: tipo}{res : dato}
 {$\neg tipoDelDato$}
 {$res \igobs datoString(n,tipoDelDato)$}
 {Crea un dato de valor de letras.}
 {O($1$)}
 {Se devuelve res por referencia.}

%%%%%% Otras Operaciones %%%%%%%%%%%

 \operacion{mismoTipo?}{in d1: dato, in d2: dato}{res:bool}
 {$true$}
 {$res \igobs mismoTipo?(d1,d2)$}
 {Informa si los datos pasados por parametro son del mismo tipo de valor.}
 {O($1$)}
 {} 
 
  \operacion{String?}{in d: dato}{res:bool}
 {$true$}
 {$res \igobs String?(d)$}
 {Informa si el dato pasado por parametro es de tipo string.}
 {O($1$)}
 {} 
 
  \operacion{Nat?}{in d: dato}{res:bool}
 {$true$}
 {$res \igobs Nat?(d)$}
 {Informa si el dato pasado por parametro es de tipo nat.}
 {O($1$)}
 {}  
 
  \operacion{min}{in cd: Conj(dato)}{res:dato}
 {$\neg$EsVacio?(cd)}
 {$res \igobs min(cd)$}
 {Retorna el minimo entre los valores del conjunto de datos pasado por parametro.}
 {O($Cardinal(cd)$)}
 {Retorna res por referencia.}

  \operacion{max}{in cd: Conj(dato)}{res:dato}
 {$\neg$EsVacio?(cd)}
 {$res \igobs max(cd)$}
 {Retorna el maximo entre los valores del conjunto de datos pasado por parametro.}
 {O($Cardinal(cd)$)}
 {Retorna res por referencia.}

  \operacion{<=}{in d1: dato, in d2: dato}{res:bool}
 {mismoTipo?(d1,d2)}
 {$res \igobs <=(d1,d2)$}
 {Retorna el maximo entre los valores del conjunto de datos pasado por parametro.}
 {O($Cardinal(cd)$)}
 {Retorna res por referencia.}
 oincidenTodos(crit,campos(crit),Siguiente(cr))
%%%%%%%%%%%%%%%%%%%%%%%%%%%%%%%%%%%%%%%%%%%%%%%%%%%%%%%%%%%%%%%%%%%%%%%%%%%%%
\subsection{Representación}

\serc{dato}{{\tupla{Valor: $\alpha$, TipoValor: bool}
  }
}

\subsubsection*{Invariante de representación}

\begin{enumerate}
  \item El Nombre de la tabla es un String acotado.
  \item Indices es un arreglo de tamaño 2, que aloja el Indice correspondiente segun el orden de creacion.
  \item Para toda Dato que es clave en Indice, su significado llamemoslo sign esta incluido en Registros.
  \item 

\end{enumerate}


\subsubsection*{Función de abstracción}
\abs{sistema}{CampusSeguro}{s}{cs}
$s.campus \igobs campus(cs)$ $\land$ \\
$s.estudiantes \igobs estudiantes(cs)$ $\land$ \\
$s.hippies \igobs hippies(cs)$ $\land$ \\
$s.agentes \igobs agentes(cs)$ $\land$ \\
$((\paratodo{nombre}{n}) s.hippies.definido(n) \impluego s.hippies.obtener(n) \igobs posEstYHippie(n,cs)$ $\lor$ \\
$(\paratodo{nombre}{n}) s.estudiantes.definido(n) \impluego s.estudiantes.obtener(n) \igobs posEstYHippie(n,cs))$ \\
$(\paratodo{placa}{pl}) s.agentes.definido(pl) \impluego s.estudiantes.obtener(pl).pos \igobs posAgente(pl,cs))$ \\
$(\paratodo{placa}{pl}) s.agentes.definido(pl) \impluego s.estudiantes.obtener(pl).cantSanciones \igobs cantSanciones(pl,cs))$ \\
$(\paratodo{placa}{pl}) s.agentes.definido(pl) \impluego s.estudiantes.obtener(pl).cantCapturas \igobs cantCapturas(pl,cs))$ \\

%%%%%%%%%%%%%%%%%%%%%%%%%%%%%%%%%%%%%%%%%%%%%%%%%%%%%%%%%%%%%%%%%%%%%%%%%%%%%%%%%%
\newpage

\subsection{Algoritmos}

\algoritmo{tipo?}{in a: dato}{res:bool}{
  \State $res \larr a.TipoValor$ \complejidad{O(1)}
}{O(1)}

\algoritmo{valorNat}{in a: dato}{res: nat}{
  \State $res \larr a.Valor$ \complejidad{O(1)}
}{O(1)}

\algoritmo{valorStr}{in a: dato}{res: string}{
  \State $res \larr a.Valor$ \complejidad{O(1)}
}{O(1)}

\algoritmo{mismoTipo?}{in d1: dato, in d2: dato}{res: bool}{ 
  \State $res \larr tipo?(d1)=tipo?(d2)$ \complejidad{O(1)}
}{O(1)}

\algoritmo{Nat?}{in a: dato}{res: bool}{
  \State $res \larr tipo?(a)$ \complejidad{O(1)}
}{O(1)}

\algoritmo{String?}{in a: dato}{res: bool}{
  \State $res \larr $\neg Nat?(a)$ \complejidad{O(1)}
}{O(1)}

   \algoritmo{min}{in cd: Conj(dato)}{res: dato}{
  	\State	itcd $\larr$ CrearItConj(cd)				
  	\State	minimo $\larr$ Siguiente(itcd)
  	\State 	\While{HaySiguiente(itcd)}
  					\State	\If {Siguiente(itcd)<=minimo} 
							  	\State	minimo $\larr$ Siguiente(itcd);
				\State	\EndIf				
				\State	Avanzar(itcr);
	\State 	\EndWhile   
 }{O(Cardinal(cd))}

   \algoritmo{max}{in cd: Conj(dato)}{res: dato}{
  	\State	itcd $\larr$ CrearItConj(cd)				
  	\State	maximo $\larr$ Siguiente(itcd)
  	\State 	\While{HaySiguiente(itcd)}
  					\State	\If {maximo<=Siguiente(itcd)} 
							  	\State	minimo $\larr$ Siguiente(itcd);
					\State	\EndIf				
				\State	Avanzar(itcr);
	\State 	\EndWhile   
 }{O(Cardinal(cd))}

   \algoritmo{<=}{in d1: dato, in d2: dato}{res: bool}{
  	\State	\If {String?(d1)} 
			  		\State	res $\larr$ valorStr(d1)<=valorStr(d2)
	\State	\Else
			  		\State	res $\larr$ valorNat(d1)<=valorNat(d2)			
	\State	\EndIf				
 }{O(1)}
 
 
  %%%%%%%%%%%%%%%%%%%%%%%%%%%%%%%%%%%%%%%%%%%%%%%%%%%%%%%%%%%%%%%%%%%%%%%%%%%%%%%%%

\subsection{Algoritmos operaciones auxiliares}
