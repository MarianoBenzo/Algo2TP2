\section{Modulo Diccionario Lexicografico($String$, $\sigma$)}

\subsection{Interfaz}

\sexc{Diccionario(String, \sigma) Iterador Bidireccional (Tupla(String , \sigma))}
\generos{diccString(String, \sigma), itDiccString(String, \sigma), itClavesString}

  \subsection*{Operaciones del diccionario}

\operacion{Vacio}{}{res : diccString(String, \sigma)}
 {$true$}
 {$res \igobs vacio()$}
 {Crea un nuevo diccionario}
 {$O(1)$}
 {}

 
 \operacion{Definido?}{in d:diccString(String, \sigma) in n:String}{res:Bool}
  {$true$}
  {$res \igobs def?(n,d)$}
  {Indica si la clave esta definida}
  {$O(Longitud(n))$}
  {}

 \operacion{Definir}{in/out d:diccString(String, \sigma) in n:String, in s:\sigma}{}
  {$d=d_0$}
  {$d \igobs Definir(n, s, d_0)$}
  {Se define s en el diccionario}
  {$O(Longitud(n))$}
  {$s$ se define por referencia. En caso de ya estar definido n, pisa la definición anterior}

 \operacion{Borrar}{in/out d:diccString(String, \sigma) in n:String}{}
 {$d \igobs d_0 \land def?(n,d)$}
 {$d \igobs Borrar(n,d_0)$}
 {Elimina el elemento n}
 {$O(Longitud(n))$}
 {}

 \operacion{Significado}{in d:diccString(String, \sigma) in n:String}{res:\sigma}
 {$def?(n,d)$}
 {$res \igobs obtener(n,d)$}
 {Se retorna el significado de n}
 {$O(Longitud(n))$}
 {}
 
 \operacion{DiccClaves}{in d:diccString(String \sigma) }{res: conj(string)}
 {$TRUE$}
 {$res \igobs claves(d)$}
 {Se retorna el conjunto de claves del diccionario}
 {$O(1)$}
 {Res un conjunto devuelto por referencia no modificable. }
 
 \operacion{Maximo}{in d:diccString(String \sigma)}{res:String}
 {$true$}
 {$res \igobs max(claves(d))$}
 {Se retorna la clave maxima en orden lexicográfico}
 {$O(longitud(k))$ donde k es la aplabra más larga del diccionario}
 {}
 
\operacion{Minimo}{in d:diccString (String \sigma)}{res:String}
 {$true$}
 {$res \igobs min(claves(d))$}
 {Se retorna la clave minima en orden lexicográfico}
 {$O(longitud(k))$ donde k es la aplabra más larga del diccionario}
 {}

 
  \subsection{Operaciones del iterador}

  El iterador que presentamos permite modificar el diccionario recorrido.  Sin embargo, cuando el diccionario es no modificable, no se pueden utilizar las funciones de eliminacion.  Ademas, las claves de los elementos iterados no pueden modificarse nunca, por cuestiones de implementacion.  Cuando $d$ es modificable, decimos que $it$ es modificable.
  

  Para simplificar la notacion, vamos a utilizar clave y significado en lugar de $\Pi_1$ y $\Pi_2$ cuando utilicemos una tupla($String,\sigma$).
 \operacion{CrearIt}{in d: diccString(String, \sigma) in n:String}{res: itDiccString($String,\sigma$)}
 {$true$}
 {alias(esPermutacion(SecuSuby($res$), $d$)) $\land$ vacia?(Anteriores($res$))}
 {crea un iterador bidireccional del diccionario, que apunta al primer elemento del mismo en orden lexicografico.}
  {$O(CL*long(k))$ Donde $CL$ es la cantidad de claves de d y $k$ la palabra mas larga de d}
  {hay aliasing entre los significados en el iterador y los del diccionario}


  \operacion{HaySiguiente}{in it:itDiccString(String,\sigma) in n:String}{res:bool}
	{$true$}  
  {$res$ $\igobs$ haySiguiente?($it$)}
  {devuelve \texttt{true} si y solo si en el iterador todavia quedan elementos para avanzar.}
  {$O(1)$}
  {}

  \operacion{HayAnterior}{in it: itDiccString(String,\sigma) in n:String}{res:bool}
	{$true$}    
  {$res$ $\igobs$ hayAnterior?($it$)}
  {devuelve \texttt{true} si y solo si en el iterador todavia quedan elementos para retroceder.}
  {$O(1)$}
  {}

  \operacion{Siguiente}{in it: itDiccString(String,\sigma) in n:String}{res:tupla(String, \sigma)}
  {HaySiguiente?($it$)}
  {alias($res$ $\igobs$ Siguiente($it$))}
  {devuelve el elemento siguiente del iterador.}
  {$O(1)$}
  {$res$.significado es modificable si y solo si $it$ es modificable, por aliasing.  En cambio, $res$.clave no es modificable.}
  
  \operacion{SiguienteClave}{in it: itDiccString(String,\sigma) in n:String}{res:String}
  {HaySiguiente?($it$)}
  {alias($res$ $\igobs$ Siguiente($it$).clave)}
  {devuelve la clave del elemento siguiente del iterador.}
  {$O(1)$}
  {$res$ no es modficable.}

\newpage
  \operacion{SiguienteSignificado}{in it: itDiccString(String,\sigma) in n:String}{res:$\sigma$}
  {HaySiguiente?($it$)}
  {alias($res$ $\igobs$ Siguiente($it$).significado)}
  {devuelve el significado del elemento siguiente del iterador.}
  {$O(1)$}
  {$res$ es modificable si y solo si $it$ es modficable, por aliasing.}

  \operacion{Anterior}{in it: itDiccString(String,\sigma) in n:String}{tupla(clave: $String$, significado: $\sigma$)}
  {HayAnterior?($it$)}
  {alias($res$ $\igobs$ Anterior($it$))}
  {devuelve el elemento anterior del iterador.}
  {$O(1)$}
  {$res$.significado es modificable si y solo si $it$ es modificable, por aliasing. En cambio, $res$.clave no es modificable.}

  \operacion{AnteriorClave}{in it: itDiccString(String,\sigma) in n:String}{res:$String$}
  {HayAnterior?($it$)}
  {alias($res$ $\igobs$ Anterior($it$).clave)}
  {devuelve la clave del elemento anterior del iterador.}
  {$O(1)$}
  {$res$ no es modficable.}

  \operacion{AnteriorSignificado}{in it: itDiccString(String,\sigma) in n:String}{res:$\sigma$}
  {HayAnterior?($it$)}
  {alias($res$ $\igobs$ Anterior($it$).significado)}
  {devuelve el significado del elemento anterior del iterador.}
  {$O(1)$}
  {$res$ es modificable si y solo si $it$ es modficable, por aliasing.}


  \operacion{Avanzar}{inout it: itDiccString(String,\sigma) in n:String}{}
  {$it = it_0$ $\land$ HaySiguiente?($it$)}
  {$it$ $\igobs$ Avanzar($it_0$)}
  {avanza a la posicion siguiente del iterador.}
  {$O(1)$}
  {}

  \operacion{Retroceder}{inout it: itDiccString(String,\sigma) in n:String}{}
  {$it = it_0$ $\land$ HayAnterior?($it$)}
  {$it$ $\igobs$ Retroceder($it_0$)}
  {retrocede a la posicion anterior del iterador.}
  {$O(1)$}
  {}


\newpage
\subsection{Representacion}
diccString\serc{dLex)}{
	\\
	\donde{dLex}{\tupla{raiz: puntero(nodo), claves: conj(string}}
}
nodo\serc{enodo)}{
	\\
	\donde{enodo}{\tupla{dato: \sigma, esSig?: Bool,  claveEnConj: itConj(String), continuaciones: puntero(char) $\sqsubset 256 \sqsupset$  }}
}

\begin{enumerate}
\item conj(string) esel Conjunto Lineal de los modulos Básicos. itConj(string) es el iterador del mismo

\end{enumerate}

\subsubsection*{Invariante de representación}


\begin{enumerate}
\item Todo Nodo, si sus continuaciones son todos punteros a null, es porque es significado.
\item No hay ciclos, ni nodos con dos padres.
\item En el conjunto claves sólo se encuentran definidas las claves del diccionario y se encuentran todas ellas
\end{enumerate}

\subsubsection*{Función de abstracción}
\tadOperacion{Abs}{diccString(string)/d}{$dicc(String \sigma)$}{Rep ($d$)}

\tadAxioma{Abs(d)}{AbsAux(d d.claves)}

\tadOperacion{AbsAux}{diccString(String )/d  conj(String)/c}{dicc(String \sigma)}{$Rep (d) \land c \subseteq d.claves $}

\tadAxioma{AbsAux(d,c)}{\IF $\emptyset ? (c)$ 
							THEN $\emptyset$ 
							ELSE $efinir(dameUno(c), significado(dameUno(c), d),AbsAux(d,sinUno(c)))$ 
						FI}
						
\subsubsection*{Respresentación del iterador}

  El iterador del diccionario lo recorre en orden lexicográfico. Los significados están por referencia.Se explica con el iterador bidireccional no modificable.
itDiccString($String$, $\sigma$)\serc{itdLex)}{
	\\
	\donde{dLex}{\tupla{claves: Lista($String$), significados: Lista($\sigma$)}}
}

  

  ~

  %\tadOperacion{Join}{secu($\alpha$)/a,secu($\beta$)/b}{secu(tupla($\alpha,\beta$))}{long($a$) $=$ long($b$)}
  %\tadAxioma{Join($a$, $b$)}{\IF vacia?($a$) THEN \secuencia{} ELSE \secuencia{$\langle${prim($a$), prim($b$)}$\rangle$}[Join(Fin($a$), Fin($b$))] FI}



\subsection{Algoritmos}
\subsubsection{Algoritmos del Diccionario}
\algoritmo{iVacio}{}{res:diccString}{
  \State $res.raiz \larr NULL$ \complejidad{O(1)}
  \State $res.claves \larr Vacio$ \complejidad{O(1)}
}{O(1)}

\algoritmo{iDefinir}{in/out d:diccString, in n:String, in s:\sigma}{}{
  
  \State $aux \larr d.raiz$     \complejidad{O(1)}   
  \State $i \larr 0$    \complejidad{O(1)}   
  \While{$i<Longitud(n)$}    \complejidad{O(Longitud(n))}
    	\If {$aux==NULL$}
    		\State $nNodo.esSig? \larr false$     \complejidad{O(1)}   
	    	\State$j \larr 0$
    		\While{$j<256$}    \complejidad{O(256)=O(1)}   
    			\State $nNodo.continuaciones[j] \larr NULL$    \complejidad{O(1)}   
	    	\EndWhile
    		\State $aux \larr \& nNodo$    \complejidad{O(1)}   
    	\EndIf
    \State $aux \larr aux.continuaciones[ord(n[i])]$    \complejidad{O(1)}   
    \EndWhile 
    \State $aux*.esSig? \larr True$    \complejidad{O(1)}  
    \If {$aux*.esSig? \neq True$}
    		\State $aux*.claveEnConj \larr AgregarRapido(d.claves, n)$  \complejidad{O(long(n))}
    	\EndIf
    \State $aux*.esSig? \larr True$    \complejidad{O(1)}   
    \State $aux*.dato \larr s$    \complejidad{O(1))}   
    \State el último paso se realiza por referencia
}{O(Longitud(n))}



\algoritmo{iDefinido?}{in/out d:diccString, in n:String}{res:bool}{
  
  \State $aux \larr d.raiz$     \complejidad{O(1)}   
  \State $i \larr 0$    \complejidad{O(1)}   
  \While{$i<(Longitud(n)-1) \land aux \neq NULL$}
    \complejidad{O(Longitud(n))}
    \State $aux \larr aux.continuaciones[ord(n[i])]$    \complejidad{O(1)}   
    \EndWhile 
    \If{$aux\neq NULL$}
		\State $res \larr aux*.esSig?$        \complejidad{O(1)}   
	\Else
		\State $res\larr false $        \complejidad{O(1)}
	\EndIf
}{O(Longitud(n))}


\newpage
\algoritmo{iSignificado}{in d:diccString, in n:String}{res:$\sigma$}{
  
  \State $aux \larr d.raiz$     \complejidad{O(1)}   
  \State $i \larr 0$    \complejidad{O(1)}   
  \While{$i < Longitud(n)$}
    \complejidad{O(Longitud(n))}
    \State $aux \larr aux.continuaciones[ord(n[i])]$    \complejidad{O(1)} 
    \State $i \larr i+1$  
   \EndWhile 
   \State $res\larr aux*.dato$        \complejidad{O(1)(es una referencia)}
}{O(Longitud(n))}



\algoritmo{iBorrar}{in/out d:diccString, in n:String}{}{
  
  \State $aux \larr d.raiz$     \complejidad{O(1)}   
  \State $i \larr 0$    \complejidad{O(1)}
  \State $pila(puntero(nodo)) p \larr Vacia()$    \complejidad{O(1)}   
  \While{$i < Longitud(n)$}    \complejidad{O(Longitud(n))}
    \State Apilar(p,aux)   \complejidad{O(1)}
    \State $aux \larr aux.continuaciones[ord(n[i])]$    \complejidad{O(1)}
    \State $i \larr i+1$  
  \EndWhile
  \State $aux*.esSig? \larr false$
  \State $BorrarSiguiente(aux*.claveEnConj)$
  \State $aux*esSig? \larr false$
  \State $i \larr i-1$
  \State $j \larr 0$
  \While{$aux*.continuaciones[j] = NULL \land j<256$} \complejidad{O(256)=O(1)}
   	\State $j \larr j+1$
  \EndWhile
  \If{$j < 256$}
   	\State $p \larr Vacia()$
  \EndIf
  \While{$\neg EsVacia?(p)$}
   		\State $j \larr 0$
   		\State $Tope(p)*.continuaciones[ord(n[i])] \larr NULL$
   		\While{$Tope(p)*.continuaciones[j] = NULL \land j<256$} \complejidad{O(256)=O(1)}
   			\State $j \larr j+1$
   		\EndWhile
   		\If{$j < 256$}
   			\State $p \larr Vacia()$
   		\Else
   			\State $Desapilar(p)$
   			\State $i \larr i-1$
   		\EndIf
   	\EndWhile
   		
}{O(Longitud(n))}

\algoritmo{iDiccClaves}{in/out d:diccString(String \sigma)}{res: $conj(String)$}{
  \State $res \larr d.claves$
}{O(1), d.claves se pasa por referencia}

\newpage
\algoritmo{Maximo}{in/out d:diccString(String \sigma)}{res: $String$}{
  \State $aux \larr d.raiz$     \complejidad{O(1)}   
  \State $res \larr Vacia()$
  \State $Bool f \larr True$
  \While{$f$}
  	\State $nat i \larr 256$
  	\While{$aux*.continuaciones[i-1] = NULL \land i>0$}
  		\State $i \larr i+1$
  	\EndWhile
  	\If{$i=0$}
  		\State $f \larr false$
  	\Else
  		\State $AgregarAtras(res, ord^-1 (i-1))$
  		\State $i \larr 0$
  	\EndIf
  \EndWhile
  		
}{O(1) d.claves se pasa por referencia}

\algoritmo{Minimo}{in/out d:diccString(String \sigma)}{res: $String$}{
  \State $aux \larr d.raiz$     \complejidad{O(1)}   
  \State $res \larr Vacia()$
  \State $Bool f \larr True$
  \While{$f$}
  	\State $nat i \larr 0$
  	\While{$aux*.continuaciones[i] = NULL \land i<256$}
  		\State $i \larr i+1$
  	\EndWhile
  	\If{$i=256$}
  		\State $f \larr false$
  	\Else
  		\State $AgregarAtras(res, ord^-1 (i))$
  		\State $i \larr 0$
  	\EndIf
  \EndWhile
  		
}{O(1) d.claves se pasa por referencia}


\subsubsection{Algoritmos del iterador}
\algoritmo{iCrearIt}{in/out d:diccString, in n:String}{res:itDiccString}{
  \State$lista(String) cs\larr Vacia()$
  \State$lista(\sigma) ss\larr Vacia()$
  \State$String n \larr Vacio()$
  \State $auxXrIt(cs,ss,n, d.raiz)$  \complejidad{O(Longitud(k)*tam(d))}
  \State$res.claves \larr cs$
  \State$res.significados \larr ss$

}{O(Longitud(k)*tam(d))}

\begin{enumerate}
\item  k es la palabra más larga definida en d y tam(d) es la cantidad de palabras definidas que hay \footnote{Si bien no nos fue posible realizar el calculo de complejidad correspondiente, el algoritmo recursivo recorre una vez cada nodo, y el total de nodos esta acotado por la sumatoria del largo de cada palabra, que a su vez está acotada por la cantidad de palabras multiplicada por la longitud de la palabra más larga, por lo que la cota propuesta a la complejidad es razonable}.

\end{enumerate}

\algoritmo{auxCrIt}{in/out cs:Lista(string) in/out ss:Lista(\sigma) in/out n:String in p:puntero(nodo)}{}{
	\If{$p \neq NULL$}
		\If{$p*.esSig?$}
			\State $AgregarAtras(cs,n)$
			\State $AgregarAtras(ss,p*.dato)$  \complejidad{O(1)}
		\EndIf
		\State $i \larr 0$
		\While{$i<256$}
			\State $AgregarAtras(n, ord^-1 (i))$
			\State $auxCrIt(cs,ss,n, p*.continuaciones[i])$
			\State $TirarUltimos(n, 1)$
		\EndWhile
	\EndIf
}{Tn desconocido}


\algoritmo{iHaySiguiente}{in/out it:itDiccString, in n:String}{res: $bool$}{
  \State $res \larr it.claves.siguiente \neq NULL$
}{O(1)}


\algoritmo{iHayAnterior}{in/out it:itDiccString, in n:String}{res: $bool$}{
  \State $res \larr it.claves.anterior \neq NULL$
}{O(1)}

%
\algoritmo{iSiguiente}{in/out it:itDiccString, in n:String}{res: tupla(string,\sigma)}{
  \State $res.clave \larr it.claves.siguiente*.dato$ \complejidad{O(1)(es una referencia)}
  \State $res.significado \larr it.dignificados.siguiente*.dato$ \complejidad{O(1)(es una referencia)}
}{O(1)}

\algoritmo{iSiguienteclave}{in/out it:itDiccString, in n:String}{res: $String$}{
  \State $res \larr it.claves.siguiente*.dato$ \complejidad{O(1)(es una referencia)}
}{O(1)}

\algoritmo{iSiguienteSignificado}{in/out it:itDiccString, in n:String}{res: $\sigma$}{
  \State $res \larr it.significados.siguiente*.dato$ \complejidad{O(1)(es una referencia)}
}{O(1)}

\algoritmo{iAnterior}{in/out it:itDiccString, in n:String}{res: tupla(string,\sigma)}{
  \State $res.clave \larr it.claves.anterior*.dato$ \complejidad{O(1)(es una referencia)}
  \State $res.significado \larr it.significados.anterior*.dato$ \complejidad{O(1)(es una referencia)}
}{O(1)}

\algoritmo{iAnteriorClave}{in/out it:itDiccString, in n:String}{res: $String$}{
  \State $res \larr it.claves.anterior*.dato$ \complejidad{O(1)(es una referencia)}
}{O(1)}

\algoritmo{iAnteriorSignificado}{in/out it:itDiccString, in n:String}{res: $\sigma$}{
  \State $res \larr it.significados.anterior*.dato$ \complejidad{O(1)(es una referencia)}
}{O(1)}

\newpage
\algoritmo{iAvanzar}{in/out it:itDiccString, in n:String}{}{
  \State $it.claves \larr it.claves.siguiente$ \complejidad{O(1)(es una referencia)}
  \State $it.significados \larr it.significados.siguiente$ \complejidad{O(1)(es una referencia)}
}{O(1)}
\algoritmo{iRetroceder}{in/out it:itDiccString, in n:String}{}{
  \State $it.claves \larr it.claves.anterior$ \complejidad{O(1)(es una referencia)}
  \State $it.significados \larr it.significados.anterior$ \complejidad{O(1)(es una referencia)}
}{O(1)}
